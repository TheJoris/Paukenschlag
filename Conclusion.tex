\section{Conclusions}

Here we present an overview of the results from the research that can be found in this thesis.

\subsection{Brownian Dynamics}
We succeeded in altering the Brownian dynamics algorithm by Morelli, such that it can be used for simulating diffusing and reactions in 2D and interactions with structures. By introducing the reaction volume, we have split-up the single step reaction of the algorithm by Morelli in to two steps. Now, in order to calculate the acceptance probability, we only need to calculate the reaction volume, instead of the much more difficult 'volume' of the old method. The algorithm is still simple because a reaction only depends on the distance between objects and an easy to calculate acceptance probability. 

Because we enforced the detailed balance relation between the association and the dissociation of a complex of two particles, the algorithm perfectly simulates the equilibrium properties of the reaction network. And, as we have shown, for sufficiently small time steps, also simulates the particle dynamics correctly. 

\subsection{The promoter site in eGFRD}
It is now possible to simulate promoter sites using Green's Function Reaction Dynamics. The promoter is represented as a point-sink on a rod, where a particle can bind with when it diffuses along this rod and glides over the sink. To be able to simulate this promoter using \GFRD, we derived a Green's function on a 1D domain with two absorbing boundaries, a sink and a unit instantaneous source for the particle starting position. We used the \GFRD\, simulations with the promoter implemented, to look at the temperal noise in the promoter occupancy and compare it with our analytical result. 

\subsection{Deriving the fundamental noise limit in transcription regulation}
We derived the fundamental lower bound on the noise in transcriptional regulation. In the limit of low concentrations, we derive an exact solution of our model of transcription factors arriving and leaving the promoter site without using any linearization. Comparing the analytical result with novel simulations done with \GFRD, shows that we can correctly predict the temperal noise in the promoter state in the whole range of average promoter occupancies. 

From our analytical result, we find how well a promoter can infer the transcription factor concentration in the cytoplasm from the promoter state, while measuring a time $T$. It turns out this noise is inversely proportional to the number of particles which bind to the promoter during the integration time $T$ if the promoter would never be occupied, times the fraction of the time the promoter is unoccupied.

The noise has an optimum when changing the average DNA residence time of a transcription factor. A higher residence time increases the concentration on the DNA, and thereby allowing for a higher binding rate with the promoter. On the other hand, due to the longer 1D diffusion, it is getting harder for a transcription factor to escape from the promoter after dissociating. Therefore, the temperal correlation increases, which contributes to the noise. We find that, as long as the transcription factor concentration increases linearly with the DNA residence time, the noise decreases.

Next, we compared our expression for the noise in a promoter, with an expression derived by Tka$\check{\rm c}$ik and Bialek who used a very different method. A key discrepancy is that the noise in Tka$\check{\rm c}$ik's expression is independent of the promoter occupancy, which is likely the result of a linearization done in his method. Furthermore, we show that we get better quantitative agreement with simulations if we represent the DNA as a cylinder instead of a line, as was done by Tka$\check{\rm c}$ik.

Furthermore, it is shown that the method by de Ronde to derive the correlation time by using a correlation function is, in the limit of low concentrations, equivalent to the method by van Zon. Van Zon calculated the correlation time earlier, by renormalizing the intrinsic association and dissociation rates of a receptor, by the average number of times a ligand rebinds with the receptor before escaping.

Finally, we find that the large time behavior of transcription factors binding and unbinding to the promoter can be described with a Poisson switch. Therefore, we can simulate a promoter site in a well mixed system with Gillespie, if we use the renomalized rates for the Poisson switch.

\newpage

\section{Possibilities for future research}
Although we put a lot of effort in enabling \GFRD\, to simulate promoter sites, we only used it in a simple configuration of a single promoter site reacting with one particle species. It would be interesting to expand this scheme such that the binding of a transcription factor to the promoter enables the expression of a protein. We can measure the fluctuations in the protein concentration, and compare results with simulations done earlier by van Zon \cite{VanZon2006}, where the promoter was represented as a spherical particle in the cytoplasm. 

More interesting would be, if the protein expression would be regulated by multiple promoter sites positioned in close proximity on the DNA. For instance, two promoter sites can enhance the expression of a single gene when both of the sites are occupied by a transcription factor. This way, a cell can regulate its protein levels using logic operations. These systems are often analyzed using mean-field theory where the particles are assumed to be uniformly distributed in space. Because transcription factors are present in low concentrations in the cell, and because of their 1D sliding along the DNA between the promoter sites, we expect that the spatio-temporal correlations have a profound effect on the operation of these logic gates. We base our expectation on the result found by Takahashi and ten Wolde \cite{Takahashi2010}, where they studied the effects of spatio-temporal correlations on the functioning of the MAPK pathway.

\newpage

\section{Acknowledgements}
First I like to thank Nils Becker for the many long (until 10 pm!) but fruitful discussions we had during my internship. First for helping me to understand the Brownian dynamics algorithm by Morelli, and having the idea for the reaction volume in the new algorithm. Also, his idea of the flux discontinuity at the promoter site, really helped with solving the Green's function of the promoter. Also, his detailed study of my derivation of the noise in the promoter site, really helped me understanding my own work. 

Next I want to thank Martijn Weherns and Laurens Bossen for being great colleagues and partners in the \GFRD\, development team! I could not have done my promoter simulations without their help in expanding the code. Laurens inexhaustible knowledge of 80's music lyrics and Martijn's humor made my internship at AMOLF  even more inspiring. 

Offcourse I thank Wiet de Ronde for sending his paper (accidentally?) of the noise in a receptor to all the members of the group instead of Pieter Rein alone. His paper is the base of the best chapter in my thesis, and his explanation of the paper helped me to quickly understand the material. Furthermore the Dutch humor and factual knowledge are also most appreciated!

I thank the other (former) group members: Thomas Sokolowski, Andrew Mugler, Mark Bosman, Filipe Tostevin, Chris Govern, Annemarieke Berkhout for the help in any form they gave. Especially Thomas for his historic knowledge in general, and the great tour we got in Saarbr\"ucken.

I thank Pieter Rein ten Wolde for reading and greatly improving my master thesis. Furthermore, i appreciated the freedom I got in finidng my own way in the research and deciding what to do in the \GFRD\, code.

Finally, I thank all the people of AMOLF for letting me have a 'professional' stay at the institute.