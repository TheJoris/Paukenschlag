\section{Conclusions}

Here we present an overview of the results from the reasearch that can be found in this thesis.

\subsection{Brownian Dynamics}
We succeeded in altering the Brownian dynamics algorithm by Morelli, such that it can be used for simulating diffusing and reactions in 2D and interactions with structures. By introducing the reaction volume, we have split-up the single step reaction of the algorithm by Morelli in to two steps. Now, in order to calculate the acceptance probability, we only need to calculate this volume, instead of the much more difficult 'volume' of the old method. The algorithm is still simple because a reaction only depends on the distance between objects and the easy to calculate acceptance probability. 

Because we enforced the detailed balance relation between the association and the dissociation of a complex of two particles, the algorithm perfectly simulates the equilibrium properties of the reaction network. And, as we have shown, for sufficiently small timesteps, also simulates the particle dynamics reasonably well. 

\subsection{The promoter site in eGFRD}
It is now possible to simulate promoter sites using Green's Function Reaction Dynamics. The promoter is represented as a point-sink on a rod, where a particle can bind with when it diffuses allong this rod and glides over it. To be able to simulate this promoter using \GFRD, we derived a Green's function for a 1D domain with two absorbing boundaries, a sink and a unit instantanious source for the particle starting position. We used the \GFRD\, simulations with the promoter implemented, to look at the temperal noise in the promoter occupancy and compare it with our analytical result. 

\subsection{Deriving the fundamental noise limit in transcription regulation}
We derived the fundamental lower bound on the noise in transcriptional regulation. In the limit of low concentrations, we derive an exact solution of our model of transcription factors ariving and leaving the promoter site without using any linearization. Comparing the analytical result with novel simulations done with \GFRD, shows that we can correctly predict the temperal noise in the promoter state in the whole range of average promoter occupancies. 

From our analytical result, we find how well a promoter can infer the transcription factor concentration in the cytoplasm from the promoter state, while measuring a time $T$. It turns out this noise is inversely proportional to the number of independent particles which bind to the promoter during the integration time $T$, times the fraction the promoter is unoccupied during that time.

The noise has an optimum when changing the average DNA residence time of a transcription factor. A lower residence time increases the concentration on the DNA, and thereby allowing for a higer binding rate with the promoter. On the other hand, due to the longer 1D diffusion, it is getting harder for a transcription factor to escape from the promoter after dissociating. Therefore, the temperal correlation increases, which contributes to the noise. We find that, as long as the transcription factor concentration increases linearly with the DNA residence time, the noise decreases.

Next, we compared our expression for the noise in a promoter, with an expression derived by Tka$\check{\rm c}$ik who used a very different method. A key discrepancy is that the noise in Tka$\check{\rm c}$ik's expression is independent of the promoter occupancy, which is likely the result of a linearization done in his method. Furthermore, we show that we get better quantative agreement with simumalions if we represent the DNA as a cylinder instead of, as was done by Tka$\check{\rm c}$ik, a line. 

Finally, is is shown that the method by de Ronde to derive the correlation time by using a correlation function for the promoter state is, in the limit of low concentrations, equivalent to the method by van Zon. Van Zon calculated the correlation time earlier, by renormalizing the intrinsic association and dissociation rates of a receptor, by the average number of times a ligand rebinds with the receptor before escaping.

\newpage

\section{Possibilities for future research}
Although we put a lot of effort in enabeling \GFRD\, to simulate promoter sites, we only used it in a simple configuration of a single promoter site reacting with one particle species. It would be intersesting to expand this scheme such that the binding of a transcription factor to the promoter enables the expression of a protein. We can measure the fluctuations in the protein concentration, and compare results with simulations done earlier by van Zon \cite{VanZon2006}, where the promoter was represented as a spherical particle in the cytoplasm. 

Even more intesting would be, if the protein expression would be regulated by multiple promoter sites positioned in close proximity on the DNA. For instance, two promoter sites can enhance the expression of a single gene when both of the sites are occupied by a transcription factor. This way, a cell can regulate its protein levels using logic operations. These systems are oftenly analysed using mean-field theory where the particles are asumed to be uniformly distributed in space. Because transcription factors are present in low concentrations in the cell, and because of their 1D sliding allong the DNA between the promoter sites, we expect that the spatio-temporal correlations have a profound effect on the operation of these logic gates. We base or expectation on the result found by Takahashi et al.  \cite{Takahashi2010}, where they studied the effects of spatio-temporal correlations on the functioning of the MAPK pathway.