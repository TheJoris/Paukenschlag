\section{Conclusions}

So, what can be concluded after reading this thesis?

\subsection{Brownian Dynamics}
By introducing the reaction volume, we have split-up the single step reaction of the old BD algorithm in to two steps. For calculating the acceptance probability we only need to calculate this volume, instead of the much more difficult 'volume` of the old method. This enables us to simulate diffusion in all dimensions and interaction with structures, as is required for \GFRD. The algorithm is still very simple because a reaction is only dependent on the distance between objects and and easy to calculate probability. While, as we have shown, simulates the dynamics reasonable well. Further more, the method enforces detailed balance, such that the average bound times come out correct. Although a reaction occurs in two steps, the performance is significantly better than the old algorithm.



\subsection{Sink domain in eGFRD}




\subsection{Fundamental noise limit}