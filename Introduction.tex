\section{Introduction}



\subsection{motivation}
Cells regulate their proteine levels by enhancing or repressing the expression of their genes. This is done by the binding of transcription factor proteins (TF) to specific sites on the DNA; the so called promoters. Because the number of TF's in a cell is typically very low ($\sim 10$), wether their is a TF present on a promoter or not varies largely over time. 

\subsubsection{research questions}
\begin{itemize}
 \item What is the noise limit of transcriptional regulation.
 \item How to simulate a promoter site in \GFRD.
 \item How to make brownian dynamics suitable for the complex geometries of \GFRD
\end{itemize}



\subsection{Introduction to eGFRD}
We want to simulate the many body problem of diffusing and reacting particles, in full spatial and temperal resolution. Because particles only interact via hard body interaction, we can apply a clever trick. We devide the domain of the full system into smaller subdomains, in such a way that they only contain one or two particles. For domains containing only one or two particles, there exist a rich literature of exact solutions in the form of Green's functions. The enhanced Green's Function Reaction Dynamics (\GFRD) algorithm sequentially propegates the subdomains using these Green's functions, and in effect propegates the whole system. \cite{VanZon2006}\cite{VanZon2005}\cite{Takahashi2010} The advantage is that, for low concentrations, a single round of the algorithm can amount to large steps in space and time. 

The system is initialized by setting the global time $t=0$, and drawing speherical domains centered around single particles. If two particles are close together however, they both go in the same domain to form a pair. We continue describing the behavour of a single-domain.

\subsubsection{single domains}


Each domain has a Green's function describing given that you started at the center of the domain ${\bf r_0}$
\begin{equation}
 p({\bf r}, \Delta t|{\bf r_0}) = \delta(\mathbf{r - r_0})
\end{equation}


Each domain has a Green's function $p({\bf r}, \Delta t|{\bf r_0})$ associated with it, describing the probability of where the particle will be after a time $\Delta t$, given that it started at ${\bf r_0}$ \cite{Carslaw1959}. A single particle can either escape through the outer boundary of it's domain, or undergo a monomolecular reaction. 


A pair domain has the extra possibility of undergoing a bimolecular reaction.

To give a rough idea of the algorithm, we continue describing how it works for a domain containing only one particle.

 To continue, we have to determine which of these events is going to take place and when it will happen. The tentative next-event time is drawn from the propensity $q_{\rm event i}(t)$ of each particular event happening. The next-event is  determined by taking the smallest tentative next-event time
\begin{equation}
 t_{\rm next\,event} = {\rm min}(t_{\rm monomolecular},\,t_{\rm escape},\,[t_{\rm bimolecular}]).
\end{equation}
First, the monomolecular reaction is modeled as a Poisson process
\begin{equation}
 q_{\rm mon}(t) = -k\, {\rm exp}(-k t).
\end{equation}
Second, the escape propensity is equal to the flux leaving the domain through the outer boundary. The outer boundary is made absorbing, meaning that each trajectory hitting this boundary will be removed from the ensemble. The total part of the ensemble still residing in the domain is given by the survival probability
\begin{equation}
 S(t|{\mathbf r_0}) = \int_V p({\bf r}, \Delta t|{\bf r_0}) \, d{\mathbf r},
 \elabel{Surv}
\end{equation}
where $V$ is the whole volume of the domain. This function is unity at $t=0$, when the particles is certain to be in the domain, after which it monotonically decays to zero. The required propensity is given by the flux $q_{\rm esc}(t)$ leaving the domain
\begin{equation}
 q_{\rm esc}(t|{\mathbf r_0}) = -\pdiff{S(t|{\mathbf r_0})}{t}.
\end{equation}

\subsubsection{pair domains}
The more complex procedure of sampleling a tentative next-event time for the bimolecular reaction in describred in the supplemntary material of \cite{Takahashi2010}.




In the full \GFRD\ algorithm, one can include rods (DNA) and planes (membranes) into the simulatin and let particles be exchanged between the bulk and these structures.







\subsubsection{single particle domain}


