%basic packages
%\usepackage[pdftex]{graphicx, color}
%\usepackage{rotating}
%\usepackage{subfigure}
\usepackage{verbatim}
\usepackage{amsmath}
\usepackage{amssymb}
\usepackage{cite}

% More complex packages
\usepackage{ifthen,calc}

\def\longrightharpoonup{\relbar\joinrel\rightharpoonup}
\def\longleftharpoondown{\leftharpoondown\joinrel\relbar}

\def\longrightleftharpoons{
  \mathop{
    \vcenter{
      \hbox{
      \ooalign{
        \raise1pt\hbox{$\longrightharpoonup\joinrel$}\crcr
	  \lower1pt\hbox{$\longleftharpoondown\joinrel$}
	  }
      }
    }
  }
}

\newcommand{\rates}[2]{\displaystyle
  \mathrel{\longrightleftharpoons^{#1\mathstrut}_{#2}}}

\newcommand{\letter}[1]{\begin{minipage}[t]{0.05\linewidth}{\vspace{0pt}\textsf{#1}}\end{minipage}\hspace*{-0.05\linewidth}}

%ifnot documentclass is revtex4 then else uncomment
% \def\affiliation#1{\gdef\@affiliation{#1}} \gdef\@affiliation{}
% \def\email#1{\gdef\@email{#1}}
% \gdef\@email{}
\newcommand{\aref}[1]{App.~\ref{app:#1}}
\newcommand{\arefs}[1]{Apps.~\ref{app:#1}}
\newcommand{\alabel}[1]{\label{app:#1}}
\newcommand{\sref}[1]{Sec.~\ref{sec:#1}}
\newcommand{\srefs}[1]{Secs.~\ref{sec:#1}}
\newcommand{\slabel}[1]{\label{sec:#1}}
\newcommand{\ssref}[1]{Par.~\ref{ssec:#1}}
\newcommand{\ssrefs}[1]{Pars.~\ref{ssec:#1}}
\newcommand{\sslabel}[1]{\label{ssec:#1}}
\newcommand{\tref}[1]{Table.~\ref{table:#1}}
\newcommand{\trefs}[1]{Tables.~\ref{table:#1}}
\newcommand{\tlabel}[1]{\label{table:#1}}
\newcommand{\fref}[1]{Fig.~\ref{fig:#1}}
\newcommand{\frefs}[1]{Figs.~\ref{fig:#1}}
\newcommand{\flabel}[1]{\label{fig:#1}}
\newcommand{\eref}[1]{Eq.~\ref{eqn:#1}}
\newcommand{\erefs}[1]{Eqs.~\ref{eqn:#1}}
\newcommand{\erefstwo}[2]{Eqs.~\ref{eqn:#1}~and~\ref{eqn:#2}}
\newcommand{\erefsrange}[2]{Eqs.~\ref{eqn:#1}-\ref{eqn:#2}}
\newcommand{\elabel}[1]{\label{eqn:#1}}

%shortened basic commands
\newcommand{\beq}{\begin{equation}}
\newcommand{\beqs}{\begin{subequations}}
\newcommand{\beqnn}{\begin{equation*}}
\newcommand{\beqn}{\begin{eqnarray}}
\newcommand{\beqnnn}{\begin{eqnarray*}}
\newcommand{\bml}{\begin{multline}}
\newcommand{\eeq}{\end{equation}}
\newcommand{\eeqs}{\end{subequations}}
\newcommand{\eeqnn}{\end{equation*}}
\newcommand{\eeqn}{\end{eqnarray}}
\newcommand{\eeqnnn}{\end{eqnarray*}}
\newcommand{\eml}{\end{multline}}
\newcommand{\pd}{\partial}
\newcommand{\lamda}{\lambda}
\newcommand{\el}{\ell}
\newcommand{\Null}{\O}
\newcommand{\eps}{\epsilon}
\newcommand{\dg}{\dagger}
\newcommand{\dagg}{\dagger}
\newcommand{\AMOLF}{FOM Institute for Atomic and Molecular Physics, \\ Science
Park 104, 1098 XG, Amsterdam}
\def\tt{\texttt}
%calligraphic letters
\newcommand{\call}[1]{ {\cal #1}}

%added commands
\newcommand{\avg}[1]{\langle{#1}\rangle}
\newcommand{\ket}[1]{|{#1}\rangle}
\newcommand{\bra}[1]{\langle{#1}|}
\newcommand{\braket}[2]{\langle{#1}\vert{#2}\rangle}
\newcommand{\ip}[2]{\langle{#1}|{#2}\rangle}
\newcommand{\br}[1]{\left ( #1\right )}
\newcommand{\abs}[1]{\left\vert{#1}\right\vert}
\newcommand{\sqbr}[1]{\left [ #1\right ]}
\newcommand{\acco}[1]{\left \lbrace #1\right \rbrace}
\newcommand{\andd}{\text{ and }}
\newcommand{\diff}[2]{\frac{d #1}{d #2}}
\newcommand{\pdiff}[2]{\frac{\pd #1}{\pd #2}}
\newcommand{\avgabs}[1]{\avg{\abs{#1}}}
\newcommand{\avgabssq}[1]{\avg{\abs{#1}^2}}
\newcommand{\erfc}[1]{{\rm erfc}\br{#1}}

\newcommand{\GFRD}{$^{\rm{e}} \rm{GFRD}$}
%\newcommand{\GFRD}{$^{\rm{e}} \mathcal{GFRD}$}
%\newcommand{\GFRD}{$^{\rm e}\mathsf{GFRD}$}
%\newcommand{\GFRD}{eGFRD}

%text in math env

%added figures
\newcommand{\capt}[1]{\vspace{0.5cm}\caption{#1}}
%correct spacing for math operators
\DeclareMathOperator{\rank}{rank}
\DeclareMathOperator{\sgn}{sgn}
%vectors
%bold
\newcommand{\vect}[1]{\boldsymbol{#1}}
%arrow
%\newcommand{\vect}[1]{\vec{#1}}
%macro
%\ifthenelse{\value{boldvector} = 1} {\newcommand{\vect}[1]{\boldsymbol{#1}}} {\newcommand{\vect}[1]{\vec{#1}}}

%matrix
\newcommand{\mat}[1]{\mathbf{#1}}
%underline
%\newcommand{\mat}[1]{\underline{\underline{{#1}}}}

%posters
\newcommand{\emailposter}[2]{\tt{#1}\tt{@}\tt{#2}}

%%%% Headers %%%%%
%removes number for subsection number
%\renewcommand*\thesubsection{\arabic{subsection}}


%% %%%%% example of if then in latex %%%%%%
%% \ifthenelse{\value{col} = 1}{\linespread{1}}{}

%% %%%%% example of while do in latex %%%%%%
%% \newcounter{mycounter}
%%   \whiledo{\value{mycounter}<X}{%
%%     \stepcounter{mycounter}%
%%     \begin{figure}[htbp]%
%%     \includegraphics[width=\textwidth,angle=-90]{figurename.pdf}%
%%     \caption{figure caption}%
%%     \end{figure}%
%%       }%
%% }

%% %%%%% complicated combined example %%%%%%%%
%% \newcounter{mycounter}
%% \newcommand*\insertfig[2]{%
%%   \setcounter{mycounter}{#1-1}%
%%   \whiledo{\value{mycounter}<#2}{%
%%     \stepcounter{mycounter}%
%%     \begin{figure}[htbp]%
%%     \includegraphics[width=\textwidth,
%%     angle=-90]{figurename.pdf}%
%%     \caption{caption name}
%%     \end{figure}%
%%     \ifthenelse{ \value{mycounter} = 5 }
%% 	       {\clearpage}{\quad}
%%   }%
%% }

