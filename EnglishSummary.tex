\begin{center}
{\center \bf Abstract}
\end{center}

The diffusive arrival of transcription factors to a promoter sets an upper bound on the precision by whichs cells can regulate their protein levels. In this thesis we derive an analytical expression for this lower limit, and expand the Green's Function Reaction Dynamics algorithm (GFRD) used for simulating reaction-diffusion systems, to verify our results.

We propose two coupled differential equations describing the full dynamics of a transcription factor diffusing in the cytoplasm and allong the DNA, with reversible exhange between the DNA and the cytoplasm. We do not linearize our equations or use any mean fealt approximation. Working in the low concentration limit, we find an exact solution for the temperal noise in the occupancy of a promoter site. 




To simulate a promoter site in GFRD, we have to take care of two things. The original Brownian dynamics 


Before we can simulate a promoter, we have to modify the original Brownian dynamics algorithm by Morelli used in GFRD. 



The interaction with the DNA turned out to be mathematically difficult using the original algorithm. By enforcing a detailed balance relation between the association of two particles in a complex and the dissociation of this complex, the algorithm correctly simulates the equilibrium quantities of a reaction network for a large range of timesteps. 

To simulate a promoter we derive a Green's function on a 1D domain with two absorbing boundaries and a unit instantanious source at the starting position of the particle. The promoter is represented with a sink, generating an outflux proportional to the intrinsic rate and the probability of being at the sink. 

Furthermore, 


GFRD also needed a new Brownian dynamics algorithm needs Brow  





Our analytical result correctly predicts the temperal noise in the promoter occupancy found in our simulations.

The result shows a minimum in the noise set by the time a transcription factor spends on the DNA. 

a competition between the concentration of transcription factors on the DNA, 


two minima in the noise caused by a competition between the average promoter occupancy, the arrival rate of transcription factors to the promoter site and enhanced temperal correlations due to 1D diffusion around the promoter. 

Furthermore, we compare our results with results found earlier by Tka$\check{\rm c}$ik. The primary difference is that his expression for the noise is independent of the occupancy of the promoter, which is likely due to linearizations needed in his method.


similar to what was done by Berg and Percell in their seminal work on the noise in the receptor state. 


The noise introduced by the diffusive arrival of transcription factors at the promoter site, sets a lower limit to the noise in the protein levels. 