\section{Introducing the promoter site in eGFRD}


\subsection{Introduction}

We want to model a promoter site in \GFRD. A promoter site is simulated as a point like sink situated on a rod. A particle diffusing along this rod, can, as it walks over the sink, either react with it or just move over it. There is a reactive flux proportional to the probability of the particle being at the sink and the rate constant $k$. Next to the sink, the full \GFRD\, domain will also contain two absorbing boundaries. The fluxes at these three points are used to determine the event type at the next event time.

\subsubsection{Formal definition of the domain}

The differential equation governing the domain for a particle diffusing in 1D with diffusion constant $D$ in the presence of a sink at $x_s$ with intrinsic association rate $k$ is
\begin{equation}
 \frac{\partial p(x,t|x_0)}{\partial t} = D \, \nabla ^2 p(x,t|x_0) - k \, \delta (x - x_s) p(x,t|x_0).
 \elabel{DE_I}
\end{equation}
Furthermore, there is an absorbing boundary at $-L_l$ (left boundary) and another at $L_r$ (right boundary),
\begin{eqnarray}
 p(-L_l,t|x_0) & = & 0 \\
 p(L_r,t|x_0) & = & 0.
\elabel{BC_I}
\end{eqnarray}
The initial condition is given by an unit instantaneous source at $x_0$
\begin{equation}
 p(x,t|x_0) = \delta(x-x_0)\delta(t)
\elabel{IC_I}
\end{equation}

\begin{figure}[bh]
\centering
\includegraphics[scale=1]{Sink_domain}
\caption{\flabel{SinkDomain} Schematic representation of the sink domain, and the used variables. The sink is situated at the origin, and the absorbing boundaries are at $-L_l$ and $L_r$. Particle starts at the position $x_0$.}
\end{figure}



\subsection{Obtaining the Green's function}

Solving equation (\eref{DE_I}) directly is difficult due to the term containing both the delta function and the function we want to solve for. We can however split the full domain $[-L_l,L_r]$ in to the sub-domains $[-L_l,x_s]$ and $[x_s,L_r]$, with the delta sink in between. The delta-sink is now introduced into the problem by the continuity conditions which have to hold between the two domains. 

\subsubsection{Deriving the continuity conditions}

The continuity equation for our problem is
\begin{equation}
 \nabla \vec{J}(x,t|x_0) + \partial_t p(x,t|x_0) = -k \delta( x - x_s ) p(x,t|x_0),
\end{equation}
which means that probability density is conserved everywhere (ignoring boundaries) except at the position of the sink $x_s$, where it `disappears' with a rate $k\,p(x_s,t|x_0)$. We can simplify this condition by first plugging in Fick's law for the probability current $\vec{J}$, and integrating over a small region $[x_s - \epsilon,x_s + \epsilon]$ around the sink, 
\setlength{\jot}{10pt}
\begin{gather}
\int_{x_s - \epsilon}^{x_s + \epsilon} \partial_x \left( -D \partial_x p(x,t|x_0)  \right) \, dx + \partial_t \int_{x_s - \epsilon}^{x_s + \epsilon} p(x,t|x_0) \, dx = \nonumber \\
-k \int_{a - \epsilon}^{a + \epsilon} \delta( x - x_s ) p(x,t|x_0) \, dx, \nonumber
\end{gather}
introducing $P$ for the primitive of p w.r.t. $x$
\begin{equation}
\left. -D \partial_x p(x,t|x_0)  \right|_{x_s - \epsilon}^{x_s + \epsilon} + \partial_t \left. P(x,t|x_0) \right|_{x_s - \epsilon}^{x_s + \epsilon} = -k p(x_s,t|x_0) \nonumber,
\end{equation}
and multiplying the second term on the left hand side with $2\epsilon/2\epsilon$, we get
\begin{gather}
J_r(x_s + \epsilon,t|x_0) - J_l(x_s - \epsilon,t|x_0) + \partial_t \frac{P(x_s + \epsilon,t|x_0) - P(x_s - \epsilon,t|x_0)}{2 \epsilon} 2\epsilon = \nonumber \\ 
-k p(x_s,t|x_0). \nonumber
\end{gather}
Now taking the limit $\epsilon \rightarrow 0$,
\begin{gather}
 \lim_{\epsilon \to 0} J_r(x_s + \epsilon,t|x_0) - J_l(x_s - \epsilon,t|x_0) + \partial_t p(x_s,t|x_0) 2\epsilon = -k p(x_s,t|x_0) \nonumber \\
 J_r(x_s,t|x_0) - J_l(x_s,t|x_0) = -k p(x_s,t|x_0).
\end{gather}
Clearly, the sink causes a discontinuity in the flux from the left to the right domain $J_l$, and vice versa for $J_r$, at the boundary. We use this result to state the continuity conditions which must hold between the two domains:
\newline

(I) The solution for the left domain $p_l$ has to equal the solution for the right domain $p_r$ at $x_s$,
\begin{equation} 
 p_l(x_s,t|x_0) = p_r(x_s,t|x_0) 
 \elabel{Cont_I}
\end{equation}.
(II) The flux from the left domain at $x_s$ should equal the flux from the right domain minus the flux which left via the sink,
\begin{equation} 
 \left. \pdiff{}{x} p_l(x,t|x_0) \right|_{x_s} = \left. \pdiff{}{x} p_r(x,t|x_0) \right|_{x_s} - \frac{k}{D} \, p_{l}(x_s,t|x_0).
 \elabel{Cont_II}
\end{equation}

\subsubsection{Ansatz equations}

The difficult problem of solving (\eref{DE_I}) has now been simplified to solving the well known heat equation in both domains. To simplify even more we let the Laplace operator act on the time domain of the heat equation. The time variable $t$ is replaced by the Laplace variable $s$ in all functions (Laplace transformed functions are denoted by a hat on top $\hat{f}$) and the time derivative in the heat equation changes in a multiplicative factor
\begin{equation}
 s \, \hat{p}(x,s|x_0) = D \, \nabla^2 \hat{p}(x,s|x_0).
 \elabel{LDEQ}
\end{equation}
Furthermore we assume the delta-sink to be positioned at the origin $x_s=0$. Together with the boundary conditions (\eref{BC_I}) and the initial condition (\eref{IC_I}), where we assume the position of the initial delta peak to be in the right domain, we can write down an Ansatz in Laplace space for the two domains \cite{Carslaw1959}. For the right domain
\begin{equation}
 \hat{p}_r(x,x_0,q) = \frac{1}{2 D q} e^{-q |x-x_0|} + A \, \mathrm{cosh} (q x) + B \, \mathrm{sinh} (q x),
 \elabel{Anz_R}
\end{equation}
and the left domain
\begin{equation}
 \hat{p}_l(x,x_0,q) = E \, \mathrm{cosh} (q x) + F \, \mathrm{sinh} (q x),
\elabel{Anz_L}
\end{equation}
where the unknowns $A,B$ and $E,F$ can be functions of $q$, with $q=\sqrt{s/D}$. The first term in the Anstatz for the right domain is the free propegator for a particle starting at $x_0$, and sets the initial condition of the problem. The hyperbolic functions are the most general solution of the diffusion equation (\eref{LDEQ}). By filling in the Ansatz into the boundary and continuity conditions we find the four unknowns. Plugging these back into the Ansatz and simplifying, the solution in Laplace space becomes
\begin{equation}
 \hat{p}_r(x,q|x_0) = -2 \, \frac{\mathrm{sinh} \, q(L_r - x)}{2 D q} \left( \frac{Dq \, \mathrm{sinh} \, q(L_l + x_0) + k \, \mathrm{sinh} \, q L_l \,\, \mathrm{sinh} \, q x_0}{Dq \, \mathrm{sinh} \, q(L_r + L_l) + k \, \mathrm{sinh} \,\, q L_l \, \mathrm{sinh} \, q L_r} \right)
 \elabel{SOL_R}
\end{equation}

\begin{equation}
 \hat{p}_l(x,q|x_0) = -\frac{\mathrm{sinh} \, q(L_l + x) \,\, \mathrm{sinh} \, q(L_r - x_0)}{Dq \, \mathrm{sinh} \, q(L_r + L_l) + k \, \mathrm{sinh} \, q L_l \,\, \mathrm{sinh} \, q L_r}
\elabel{SOL_L}
\end{equation}

\subsubsection{Transforming the Laplace solution to the time domain}

In order to transform our solution in Laplace space back to the time domain, we have to solve the Bromwich integral
\begin{equation}
 \frac{1}{2 \pi i} \int_{-\infty i + \gamma}^{+\infty i + \gamma} \, e^{s t} \, \hat{p}(x,s|x_0) \, ds.
 \elabel{INT_I}
\end{equation}
Our solution in Laplace space is not of a simple form which can be looked up in a transform table, neither can we use the primitive of the integrand and fill in the integration boundaries; finding the primitive is beyond our capabilities. Therefore we have to reside to residual integration. We will only roughly sketch the steps needed in residual integration; more complete and excellent accounts are given in \cite{Carslaw1959} and \cite{Bossen2011a}.

When we integrate function $f$ of a complex variable $z$ counter-clockwise around a closed contour $\gamma$, this is equal to the sum over all residuals at positions $z = c_i$ of $f$ within $\gamma$
\begin{equation}
 2 \pi i\,\sum_{i=1}^{N}\,\mathrm{Res}(f,c_i) = \oint_\gamma f(z)\,dz.
 \elabel{RES_I}
\end{equation}

Investigating our solutions \erefstwo{SOL_R}{SOL_L}, we find that they are rational functions where both the numerator and denominator are holomorphic. We denote the numerator by $g_{r/l}(x,q|x_0)$ and the denominator by $h(q)$, where the subscript $r/l$ of the numerator indicates the solution for the right or the left domain. Both solutions have the same denominator apart from the factor $2 D q$ in the right domain, which we ignore for now because it cancels later on. Our solutions $p_{r/l}$ can thus be written as
\begin{equation} 
 \hat{p}_{r/l}(x,q|x_0) = \frac{g_{r/l}(x,q|x_0)}{h(q)}.
\end{equation}
Assuming our solutions have simple poles, from the theory of complex integration, we know that the residues of these functions can be found via
\begin{equation} 
 \mathrm{Res} \left( \frac{g_{r/l}(x,q|x_0)}{h(q)},\beta_i \right) = \left. \frac{g_{r/l}(x,q|x_0)}{\partial_q h(q)} \right|_{q = \beta_i},
 \elabel{RES_II}
\end{equation}
where the $\beta_i$ 's are the positions of the simple poles and are the roots of the denominator $h$,
\begin{equation} 
 Dq \, \mathrm{sinh} \, q (L_r + L_l) + k \, \mathrm{sinh} \, q L_l \, \mathrm{sinh} \, q L_r = 0.
 \elabel{ROOT_I}
\end{equation}

The above equation only holds for real $q$ when $q = 0$, but this root can be ignored because $g(x,q=0|x_0) = 0$. However, the equation has an infinite number of solutions when $q$ is purely imaginary. Remember that $q = \sqrt{s/D}$, which means that there are only poles for real $s < 0$. Looking at \eref{INT_I}, the negative $s$ will give a solution in the time domain which is monotonically decaying. This is obvious, because for $t > 0$ we only have sinks and no source terms. Hence, from now on, we assume $q$ to be of the form $q = i \bar{q}, \, \bar{q} \in \mathbb{R}_{>0}$.

If we first apply the transformation $q \rightarrow i \bar{q}$ and than use \erefstwo{RES_I}{RES_II}, the integral in \eref{INT_I} becomes
\begin{equation} 
 \frac{1}{2 \pi i} \oint_{\gamma}
 \, e^{- D \bar{q}^2 t} \, \frac{g(x,i \bar{q}|x_0)}{h(i \bar{q})} \, 2 D i \bar{q} \, d\bar{q} = \sum_{i=1}^{\infty} e^{ - D \bar{q}^2 t} \, \left. \frac{g(x,i \bar{q}|x_0)}{\partial_q h(i \bar{q})} 2 D i \bar{q} \right|_{\bar{q} = \beta_i},
\end{equation}
where we used that $ds = 2Dq\,dq$. Now we can fill in our Laplace solutions and obtain the final solution in real space. The hyperbolic functions will become trigonometric functions due to the purely imaginary arguments. Collecting all imaginary terms, and using the fact that the function is evaluated at the roots of \eref{ROOT_I}, we find that indeed the imaginary part disappears and are left with a real function. Writing the result such that it is manifestly zero at the boundaries, we find
\begin{multline}
 p_r(x,t|x_0) = -2 \sum_{i=1}^{\infty} \, e^{- D \beta_i ^2 t} \, \mathrm{sin} \, \beta_i (L_r - x) \\ 
\frac{D \beta_i \, \mathrm{sin} \, \beta_i (L_l + x_0) + k \, \mathrm{sin} \, \beta_i L_l \,\, \mathrm{sin} \, \beta_i x_0}{D \, (L \beta_i \mathrm{cos} \, \beta_i L + \, \mathrm{sin} \, L \beta_i ) + k \,( L_r \, \mathrm{cos} \, \beta_i L_r \,\, \mathrm{sin} \, \beta_i L_l + L_l \, \mathrm{cos} \, \beta_i L_l \,\, \mathrm{sin} \, \beta_i L_r )},
 \elabel{FSOL_R}
\end{multline}

\begin{multline}
 p_l(x,t,|x_0) = -\sum_{i=1}^{\infty} e^{- D \beta_i ^2 t} \, 2 D \beta_i \, \\ 
\frac{\mathrm{sin} \, \beta_i (L_l + x) \,\, \mathrm{sin} \, \beta_i (L_r - x_0)}{D \, (L \beta_i \mathrm{cos} \, \beta_i L + \, \mathrm{sin} \, L \beta_i ) + k \,( L_r \, \mathrm{cos} \, \beta_i L_r \,\, \mathrm{sin} \, \beta_i L_l + L_l \, \mathrm{cos} \, \beta_i L_l \,\, \mathrm{sin} \, \beta_i L_r )},
\elabel{FSOL_L}
\end{multline}
where $L=L_r + L_l$ in both equations. Note that when $p_r$ is evaluated for $x<x_0$, $x$ and $x_0$ have to be interchanged in equation (\eref{FSOL_R}). This is due to the absolute sign in the Ansatz (\eref{Anz_R}). In \aref{sink:GFRD} we give an oversight of how these results are used in \GFRD.


\subsection{Validating the solution}
To validate our solution, we first check if it obeys the imposed boundary conditions. Clearly, \erefstwo{FSOL_R}{FSOL_L} are zero at $L_r$ and $-L_l$ respectively, because each term in the solution is proportional to a sinus function which equals zero at these positions. To check the continuity condition $p_r(x=0,t|x_0)=p_l(x=0,t|x_0)$, we interchange $x$ with $x_0$ in $p_r$ and evaluate both solutions at $x=0$. Only comparing numerators, the term proportional to $k$ vanishes in $p_r$, and both equations become the same. 

The most important condition is that the flux imbalance at the position of the sink is proportional to the probability density at that point, \eref{Cont_II}. To check, we differentiate both the numerators with respect to $x$, let $x=0$, and subtract them to get
\begin{equation}
 2 D s^2 \mathrm{cos}[L_l s] \, \mathrm{sin}[s (L_r-x_0)]-2 \Big(D s^2 \mathrm{cos}[L_l s]+k s \mathrm{sin}[L_l s]\Big) \mathrm{sin}[s (L_r-x_0)].
\end{equation}
The first and second term in this equation cancel, leaving the third term which is equal to $k/D \, p_l(x=0,t|x_0)$, as was required. 

To validate it's implementation in \GFRD\,, we do a simple simulation with one particle and one sink on a finite rod. The particle can bind with the sink with intrinsic rate $k_+$. When the particle is bound with the promoter, it can unbind from it with an intrinsic rate $k_-$. We measure the time the particle was bound with the sink, and calculate the average occupancy. This we can check with the analytical result for an equilibrated system,
\begin{equation}
\bar{n} = \frac{k_+ c}{k_+ c + k_-}.
\end{equation}
The result in \fref{sink_bound} shows excellent agreement with the theory.

\begin{figure}[th]
\centering
\includegraphics[scale=.5]{SinkPbound}
\caption{\flabel{sink_bound} Average occupancy of the sink for different values of the intrinsic association rate $k$. The domain size is $L=1\mu m$, containing one particle. The dissociation rate is $k_-=1ms$. Error bars are smaller than the plot dots.}
\end{figure}

\subsection{Visualizing the Green's function}
To illustrate how the Green's function behaves, we show in \fref{sink} how the probability distribution (\erefstwo{FSOL_R}{FSOL_L}) and the flux changes over time. The left figure clearly shows a discontinuity at the position of the sink, and is zero at the boundaries. For small times, the probability distribution closely resembles that of the free propagator in an unbounded domain, because little flux has left the domain yet.

\begin{figure}[ht]
\begin{minipage}[ht]{.5\linewidth}
\centering
\includegraphics[scale=.3]{AbsSinkAbs_P}
\end{minipage}
\begin{minipage}[ht]{.5\linewidth}
\centering
\includegraphics[scale=.3]{AbsSinkAbs_Flux}
\end{minipage}
\caption{\flabel{sink} Green's function in a 1D domain with two absorbing boundaries at $r=0$ and $r=13$, a sink with rate $k=1$ at $x_s=5$ and a unit instantaneous source at $x_0=7$. The diffusion constant is $D=1$. {\bf Left}: Change of the probability distribution over time between $t=1$ (blue, on top) to $t=10$ (red, on the bottom). The discontinuity at the position of the sink is manifest. {\bf Right}: Fluxes leaving the domain. The total flux (blue line) is all the probability leaving the domain at time $t$. Because the particle starts closest to the sink, the flux through the sink (green) rises first. At later times the distribution starts to `see' the right boundary, as is shown of the left plot, and the flux through this (red) boundary starts to rise. The flux at the left boundary (light blue) is negative because probability leaves the domain flowing to the left. 
}
\end{figure}

\subsection{Solution for a sink in an unbounded domain}
For completes and its usefulness in \GFRD, we end this derivation with the Green's function for an absorbing sink at $x_s$ in an infinite domain. The solution in Laplace space reads
\begin{equation}
 \hat{p}(z,s|\sigma) = \frac{1}{2 D_1 q}e^{-q|z|} - \frac{k_+}{4 D^2}\frac{e^{-q|z|}}{q(q+\frac{k_+}{2 D_1})}, \quad q=\sqrt{\frac{s}{D_1}},
 \elabel{LSOL_NB}
\end{equation}
which we can transform to the time domain using a table in \cite{Carslaw1959}
\begin{multline}
 p(x,t|x_0) = \frac{1}{\sqrt{4 \pi D t}}e^{\frac{-(x-x_0)^2}{4 D t}} - \frac{k}{4 D}e^{\frac{k}{2 D}(|x-x_s|+|x_s-x_0|)+\frac{k^2 t}{4 D}} \\
\mathrm{erfc}\left( \frac{|x-x_s|+|x_s-x_0|}{\sqrt{4 D t}}+\frac{k}{2D}\sqrt{D t} \right).
\elabel{FSOL_NB}
\end{multline}
We can map this solution to the solution of a radiation boundary condition at $x_s$ in a semi-infinite domain when we change the free propegator in the first term with the propegator for a reflective wall at $x_s$. Furthermore, the second term has to be multiplied with 2 and the rate $k$ needs to be doubled $k \rightarrow 2 k$. Intuitively, the probability for a particle to be at $x_s$ is two times as large in the semi-infinite domain, as it is in the full domain, because the solution of the full domain is mirrored around $x_s$. Therefore, the outflux is also two times as large, and we have to double the reaction rate and the second term in \eref{FSOL_NB} to account for this.

The survival probability function $S(t|x_0)$ of the above function is equal to that of a semi-infinite system with a radiation boundary at $x_s$ when we double the rate $k$.