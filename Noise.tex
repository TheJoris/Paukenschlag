\section{The fundamental limit on noise in transcriptional regulation}

\subsection{Introduction}
As was motivated in the introduction, the time a promoter is occupied by a transcription factor can vary strong over time. We want to find out how this variance depends on the parameters such as the binding affinities to the promoter and the DNA, the decay rate from the DNA and the diffusion constants. We define our variance as follows. We measure the time a promoter site is occupied over a time $T$, where $T$ is much larger than the correlation time $\tau_c$. If we repeat this measurement many times, the variance in our collection of measurements is given by (\aref{noise:corr_time})
\begin{equation}
 \left( \delta n \right)^2 \equiv \sigma_{\rm n, T}^2 = \frac{2 \sigma_{\rm n}^2 \,\tau_{\rm c}}{T}
 \elabel{std_devT}
\end{equation}
where $\sigma_n^2$ is the variance of an instantaneous measurement. The variance of an instantanious measurement for a binomial process (the promoter has only two states) is known to be $\bar{n}(1-\bar{n})$, with $\bar{n}$ the average occupancy. Therefore we only need to find the correlation time. 

We will first derive the correlation function, from which we can obtain the correlation time. With this correlation time in hand, we obtain general relations for the noise in the promoter occupancy, concentration fluctuations around the promoter and the lower limit set by the diffusive arrival. We continue with obtaining expressions in different geometries. First for a system where particles can only move on the DNA (1D only), secondly for DNA in a well mixed solution and end with a system where TF's can perform multiple rebinding with the DNA before binding to the promoter.


\subsection{Methods}

\subsubsection{The correlation function}
Here we derive the auto-correlation function for a switching process in Laplace space as was done by de Ronde et al. \cite{DeRonde2012}. We start with writing down the general expression for the correlation function of a switch
\begin{eqnarray}
 C(\tau) & \equiv & \left\langle (n(\tau) - \bar{n})(n(0) - \bar{n}) \right\rangle \\
  	& = & \bar{n} \left( p_{*|*}(\tau)-\bar{n} \right).
 \elabel{corr_func}
\end{eqnarray}
The function $n(t)$ depicts the promoter state and is either one when the promoter is occupied and zero when it is free. In the first line we used that, for a stationary system, we can shift the time such that only one function depends on time. In the second line we introduced the condinional probability that given the promoter was bound at $t=0$, what is the probability that it is bound at a later time $\tau$. This conditional probability is equal to
\begin{equation}
 p_{*|*} = 1 - \mathscr{S}_{\rm rev}(\tau|*)
\end{equation}
where $\mathscr{S}(t|*)$ is the reversible survival probability of a free promoter given that it started in the bound state. The promoter can undergo multiple rounds of binding and unbinding during the time $\tau$. We can describe this revirsible process in term of an irreversible one via the convolution \cite{Agmon1990}
\begin{equation}
 \mathscr{S}_{\rm rev}(t|*) = k_- \int_0^t [1-\mathscr{S}_{\rm rev}(t'|*)]\mathscr{S}_{\rm irr}(t-t'|\sigma)dt'.
 \elabel{rev_conv}
\end{equation}
The first factor under the integral gives the probability that the promoter is occupied at time $t'$. After which it decays for the last time with a rate $k_-$, and remains unoccupied until $t$ with a probability $\mathscr{S}_{\rm irr}(t-t'|\sigma)$. After decay, a TF is placed at contact (denoted by $\sigma$ in $\mathscr{S}_{\rm irr}$) with the promoter. Integrating over all intermediate times $t'$, gives us the probability that a promoter in unoccupied at time $t$. 

For the irreversible survival probability with one TF at contact with the promoter we propose the form
\begin{equation}
 \mathscr{S}_{\rm irr}(t|\sigma) = \mathscr{S}_{\rm irr}(t|{\rm eq}) S(t|\sigma)
 \elabel{Def_Ssi}
\end{equation}
where $\mathscr{S}_{\rm irr}(t|{\rm eq})$ is the survival probability of a free promoter site surrounded by an equilibrated sollution of TF's. $S(t|\sigma)$ is the survival probability with just one TF placed on the promoter site. This form implies that after a TF has dissociated from the promoter, all the other TF's are equilibrated. This is a good approximation when the average decay time $1/k_-$ is long enough for another particle decayed earlier to equilibrate. 

$\mathscr{S}_{\rm irr}(t|{\rm eq})$ can be found by solving the differential equation
\begin{equation}
 \pdiff{\mathscr{S}_{\rm irr}(t|{\rm eq})}{t} = - \bar{\xi} \, k(t) \, \mathscr{S}_{\rm irr}(t|{\rm eq}).
 \elabel{DE_Seq}
\end{equation}
which states that the rate at which TF's bind to the promoter is equal to the rate $k_{{\rm irr}}(t)$ at with they would bind to a sink (which is never occupied) times the probability that the promoter is still unoccupied. $\bar{\xi}$ is the average concentration of TF's on the DNA. Solving the equation yields
\begin{equation}
 \mathscr{S}_{\rm irr}(t|{\rm eq}) = e^{-\bar{\xi}\int_0^t k_{{\rm irr}}(t') dt'},
 \elabel{Seq}
\end{equation}
and $k_{{\rm irr}}(t)$ is, using the backward Smoluchowski equation, equal to (\aref{noise:backward_eqn})
\begin{equation}
 k_{\rm rad}(t) = k_+ \, S_{\rm rad}(t|\sigma).
 \elabel{krad}
\end{equation}
Before diving into the Laplace domain, we give a relation which will prove to be very usefull. Namely, from \erefstwo{DE_Seq}{krad}, it is clear that
\begin{eqnarray}
 \pdiff{\mathscr{S}_{\rm irr}(t|{\rm eq})}{t} & = & -\bar{\xi} k_+ \, S_{\rm rad}(t|\sigma) \, \mathscr{S}_{\rm irr}(t|{\rm eq}) \\
 & = & -\bar{\xi} k_+ \, \mathscr{S}_{\rm irr}(t|\sigma).
 \elabel{SeqSsi}
\end{eqnarray}

We start by Laplace transforming \eref{rev_conv} and solve it for $\mathscr{S}_{\rm rev}$. By using the Laplace transformed \erefstwo{corr_func}{SeqSsi}, we can express the correlation function as a function of $\hat{\mathscr{S}}_{\rm irr}(s|{\rm eq})$ only
\begin{equation}
 \hat{C_n}(s) = \sigma_n^2 \, \frac{\bar{n} \hat{\mathscr{S}}_{\rm irr}(s|eq)}{1-(1-\bar{n})s\hat{\mathscr{S}}_{\rm irr}(s|eq)}.
 \elabel{LapCorr}
\end{equation}
Since we can not directly Laplace transform the exponential in \eref{Seq}, we hypothesize the following form as an approximation (\aref{noise:Seq_approx})
\begin{equation}
 \lim_{c\to0} \hat{\mathscr{S}}_{\rm rad}(s|eq) = \frac{1}{s}  \, \frac{1}{ 1 + \bar{\xi} \hat{k}_{\rm rad}(s)}
 \elabel{Seq_approx}
\end{equation}
Note that for a constant rate, $\hat{k}_{\rm rad}(s) \propto k/s$, the above gives an exponential decay in the time domain, which is the exact sollution. 

Substituting this approximation into \eref{LapCorr}, we obtain after simplifying
\begin{equation}
 \lim_{c \to 0} \hat{C_n}(s) = \frac{\bar{n} \sigma_n^2}{\bar{n} s + k_+ \bar{\xi} \, s \hat{S}(s|\sigma)}.
\end{equation}
The correlation time of the process is found by taking the $s \to 0$ limit
\begin{eqnarray}
 \lim_{s \to 0} \hat{C_n}(s) = \int_0^\infty C_n(t) dt = \sigma_n^2 \, \tau_c.
\end{eqnarray}
Using the approximation in \eref{Seq_approx}, the general expression for the correlation time becomes (\aref{noise:limits_Laplace})
\begin{equation}
 \lim_{c \to 0, s \to 0} \hat{C_n}(s) = \frac{\bar{n} \sigma_n^2}{k_+ \bar{c} \, S(\infty|\sigma)} = \frac{\sigma_n^2 \tau_c^*}{S(\infty|\sigma)},
 \elabel{CT1}
\end{equation}
where $\tau_c^*$ is the correlation time for the fast proces, $\tau_c^* = k_+ \hat{c} + k_-$. To summarize, given the large time limit of $S(t|\sigma)$, we have the correlation time.


\subsubsection{Defining noise}
We define the noise in the receptor state, using \erefstwo{std_devT}{CT1}, as
\begin{equation}
 \left( \frac{\delta n}{\bar{n}} \right)^2 = 2 \, \frac{\sigma_n^2}{\bar{n}^2} \, \frac{\tau_c}{T} = 2 \, \frac{\sigma_n^2}{\bar{n}^2} \, \frac{\tau_c^*}{S(\infty|\sigma)} \, \frac{1}{T}.
 \elabel{EqProNoise}
\end{equation}
It turns out that the escape probability and $\tau_c$ have the general form
\begin{equation}
 S(\infty|\sigma) = \frac{k_D}{k_+ + k_D} \quad {\rm and} \quad \tau_c = \frac{k_+ + k_D}{(k_+ \bar{\xi} + k_-) k_D},
 \elabel{Sinfty}
\end{equation}
where $k_+$ and $k_-$ are the intrinsic association and dissociation rates of the promoter site respectively, and $k_D$ the rate at which TF's arive at an 'absorbing' promoter in steady-state. Combinig this with \eref{EqProNoise}, \eref{EqxiNoise} and \eref{EqcNoise}, we arive at general relations for the noise in the receptor
\begin{eqnarray}
 \elabel{NoisePro}
 \left( \frac{\delta n}{\bar{n}} \right)^2 & = & 2 \bar{n}(1-\bar{n})\left[\left( \frac{1}{\bar{n} k_D T \bar{\xi}} \right)+\frac{1-\bar{n}}{k_- T \bar{n}^2} \right] \\
 & = & \frac{2 \sigma_n^2}{T \bar{\xi} \bar{n}} \, \frac{1}{k_{\rm on}},
\end{eqnarray}
the noise in the concentration on the DNA at the position of the promoter
\begin{eqnarray}
 \elabel{NoiseXi}
 \left( \frac{\delta \xi}{\bar{\xi}} \right)^2 & = & \frac{2}{\bar{n}(1-\bar{n})} \left( \frac{\bar{n}}{k_D T \bar{\xi}} + \frac{1-\bar{n}}{k_- T}  \right) \\
  & = & \frac{1}{T \bar{\xi} (1 - \bar{n})} \, \frac{1}{k_{\rm on}},
\end{eqnarray}
and the noise in the 'equivalent concentration' of the cytoplasm $c$
\begin{equation}
 \elabel{NoiseC} 
 \left( \frac{\delta c}{\bar{c}} \right)^2 = \frac{2}{\bar{n}(1-\bar{n})} \left( \frac{k_d}{k_a} \, \frac{\bar{n}}{k_D T \bar{c}} + \frac{1-\bar{n}}{k_- T}  \right).
\end{equation}
We can find the lower bound for the noise, or the noise due to diffusion only, by removing all the terms having a $k_-$ in \erefsrange{NoisePro}{NoiseC}. 

For comparison with the paper of Tka$\check{\rm c}$ik \cite{Tkacik2009} we write the low $\omega$ limit of the power spectrum (\aref{noise:corr_time}) as
\begin{equation}
 S_n(\omega \to 0) = \frac{2 \bar{n}(1-\bar{n})^2}{k_-} + \frac{2\bar{n}^2(1-\bar{n})}{\bar{\xi} k_D}.
 \elabel{Sn}
\end{equation}

For some processes, such as diffusion along a line or on a plane, the correlation time diverges to infinity, and the promoter won't be able to make true independent measurements. As a result, the noise perceived by a promoter is infinite given the definition in \eref{EqProNoise}. Therefore, for these kind of processes, we adopt a different definition
\begin{equation}
  \left( \frac{\delta n}{\bar{n}} \right)^2 = \left. \frac{1}{\bar{n}^2} \lim_{t \to \infty} C_n(t)  \right|_{t = T}.
  \elabel{EqProNoiseII}
\end{equation}
Here, the noise decays as a function of the integration time $T$, as the long time behaviour of the correlation function.


\subsection{Noise in different environments}

\subsubsection{1D diffusion only}
We start by calculating the noise in a system where TF's diffuse allong the DNA only: no excursions in the cytoplasm are allowed. It is well known \cite{Redner2001} that for such a 1D system, particles allways return to their starting position. As a result, TF's are unable to 'escape' from binding to the promotor and the survival probability of the single particle problem tends to zero. Therefore, we use \eref{EqProNoiseII} for defining the noise, and we start by deriving the large time limit of the correlation function.

The survival probability in Laplace space for this system is
\begin{equation}
 \hat{S}(s|\sigma) = \frac{1}{s}\left( \frac{\sqrt{4 D_1 s}}{\sqrt{4 D_1 s} + k_+} \right),
 \elabel{LSx0}
\end{equation}
where $D_1$ is the diffusion contant in 1D and $k_+$ is the association rate to the promoter. Using the approximation for $\hat{\mathscr{S}}_{\rm irr}(s|{\rm eq})$, \eref{Seq}, and plugging this into the correlation function \eref{LapCorr}
\begin{equation}
  \hat{C_n}(s) = \sigma_n^2 \, \frac{\bar{n} \, \left( k_+ + \sqrt{4 D_1 s} \right)}{k_+ \bar{n} s + \bar{\xi} k_+ \sqrt{4 D_1 s} + \bar{n} s \sqrt{4 D_1 s}}
  \elabel{LapCorr1D}
\end{equation}
It is clear that in the limit $s \to 0$, the correlation function diverges to $+\infty$ such that $\tau_c$ is infinite. To lowest order, the expansion in $s$ reads
\begin{equation}
 \lim_{s \to 0} \hat{C_n}(s) = \frac{\bar{n}^2(1-\bar{n})}{\bar{\xi} \sqrt{4 D_1 s}}.
\end{equation}
Transforming this back to the time domain, we find the limit for large $t$
\begin{equation}
 \lim_{t \to \infty} \hat{C_n}(t) = \frac{\bar{n}^2(1-\bar{n})}{\bar{\xi} \sqrt{4 \pi D_1 t}}.
\end{equation}
The correlation function of the concentration in the large time limit is
\begin{equation}
 \lim_{t \to \infty} \hat{C_{\xi}}(t) = \lim_{t \to \infty} \left| \frac{\partial n}{\partial \xi} \right|^{-2} \hat{C_n}(t) = \frac{\bar{\xi}}{(1 - \bar{n}) \sqrt{4 \pi D_1 t}}.
\end{equation}
Using the alternative definition of the noise in \eref{EqProNoiseII}, we find
\begin{equation}
 \left( \frac{\delta n}{\bar{n}} \right)^2 =  \frac{(1-\bar{n})}{\bar{\xi} \sqrt{4 \pi D_1 t}} \quad \text{and} \quad \left( \frac{\delta \xi}{\bar{\xi}} \right)^2 = \frac{1}{ \bar{\xi} (1 - \bar{n}) \sqrt{4 \pi D_1 t}}. 
 \elabel{Noise1D}
\end{equation}
We see that the noise decreases only with the square root of the measurement time. This is due to the fact that a TF never escapes from the promoter, and it is therefore impossible for the promoter to take truly indepenent measurements of the TF concentration. 


\subsubsection{1D diffusion with exchange to a perfectly mixed bulk}
Here we imagine a system where TF's diffuse along an infinite strand of DNA and can fall off with a uniform rate $k_d$. When they fall off they are placed at a random position in the bulk. The survival probability for a TF initially at the promoter site is given by \eref{LSx0}, where, due to the uniform decay, the Laplace variable $s$ is shifted by $k_d$: $q \to \sqrt{(s+k_d)/D}$. The required long time limit of the survival probability is
\begin{equation}
 \lim_{s\to0} s \, \hat{S}(s|\sigma) = S(\infty|\sigma) = \frac{\sqrt{4 D_1 k_d}}{\sqrt{4 D_1 k_d} + k_+}.
\end{equation}
Comparing this with \eref{Sinfty}, it is clear that $k_D = \sqrt{4 D_1 k_d}$ for this system. Using the noise \erefsrange{NoisePro}{NoiseC}, we find that the noise in the promoter state is
\begin{equation}
 \elabel{NoiseProWM}
 \left( \frac{\delta n}{\bar{n}} \right)^2 = 2 \bar{n}(1-\bar{n})\left[\left( \frac{1}{\bar{n} \sqrt{4 D_1 k_d} T \bar{\xi}} \right)+\frac{1-\bar{n}}{k_- T \bar{n}^2} \right],
\end{equation}
and the noise in the 'equivalent concentration' of the cytoplasm $c$
\begin{equation}
 \elabel{NoiseCWM} 
 \left( \frac{\delta c}{\bar{c}} \right)^2 = \frac{2}{\bar{n}(1-\bar{n})} \left( \frac{1}{k_a} \, \frac{\bar{n}}{\sqrt{4 D_1 k_d^{-1}} T \bar{c}} + \frac{1-\bar{n}}{k_- T}  \right).
\end{equation}


\subsubsection{1D diffusion with exchange to the bulk}
Now we want to evaluate the noise in the occupancy of a promoter site when particles are not placed at a random position after falling from the DNA, but are placed at contact with it. A TF starting at the promoter site switches between sliding along the DNA and making excursions in the cytoplasm. The TF slides with a small diffusion constant $D_1$ and has the possibility of irreversibly binding to the promoter. After falling from the DNA, the TF diffuses around in the cytoplasm with a larger diffusion constant $D_3$, and, because it is essentially a 2D problem, will always rebinds with the DNA. Our derivation of the noise is very similar to that followed by Tka$\check{\rm c}$ik \cite{Tkacik2009}.

To find the survival probability for this problem, we start with writing down the differential equations describing the full system
\setlength{\jot}{12pt}
\begin{eqnarray*}
 \frac{\partial \xi}{\partial t} & = & D_1 \frac{\partial^2\xi(z,t)}{\partial^2 z} - k_+ \xi(z,t) \delta(z - z_0) - k_d \xi(z,t) + k_a c(z, |{\bf r}| = \sigma, t) \\
 \frac{\partial c}{\partial t} & = & D_3 \nabla^2 c(z, \mathbf r \mathrm, t) - \left[ k_a c(z, \bf r \it, t) - k_d \xi(z,t) \right] \frac{\delta(|\mathbf r \mathrm| - \sigma)}{2 \pi \sigma}.
\end{eqnarray*}
Here $\xi$(z, t) is the probability density of the TF sliding along the DNA, which we model as an infinite long rod along the z-axis with a radius $\sigma$. Probability flows irreversibly into the sink, positioned at $z_0$, with a rate $k_+$. The TF can fall from the DNA into the cytoplasm with a uniform rate $k_d$, and is placed at a distance $\sigma$ from the DNA. The particle can rebind again with a rate $k_a$. Excursions in the cytoplasm are described by $c$(z, {\bf r}, t). First we Laplace transform with respect to time
\begin{eqnarray*}
 s \, \hat{\xi} - \delta(z - z_0) & = & D_1 \frac{\partial^2\hat{\xi}(z,s)}{\partial^2 z} - k_+ \hat{\xi}(z,s) \delta(z - z_0) - k_d \hat{\xi}(z,s) + k_a \hat{c}(z, |{\bf r}| = \sigma, s) \\
 s \, \hat{c} & = & D_3 \nabla^2 \hat{c}(z, \mathbf r \mathrm, s) - \left[ k_a \hat{c}(z, \mathbf r \mathrm, s) - k_d \hat{\xi}(z,s) \right] \frac{\delta(|\mathbf r \mathrm| - \sigma)}{2 \pi \sigma},
\end{eqnarray*}
where we explicitly included the initial condition of one particle placed in contact with the promoter site on the DNA. Now we Fourier transform with respect to space (\aref{noise:delta_int})
\begin{eqnarray}
 s \, \tilde{\xi}(q, s) - 1 & = & - D_1 q^2 \tilde{\xi}(q,s) - k_+ \tilde{\xi}(z_0,s) - k_d \tilde{\xi}(q,s) + k_a \tilde{c}(q, s) \elabel{FDE_dna} \\
 s \, \tilde{c} & = & - D_3 (q^2 + k^2) \tilde{c}(q, \mathbf r \mathrm, s) - \left[ k_a \tilde{c}(q, s) - k_d \tilde{\xi}(q,s) \right] J_{0}(k \sigma). 
 \elabel{FDE_bulk}
\end{eqnarray}
Here $q$ is the spatial Fourier variable conjugate to $z$, and $\bf k \rm$ is conjugate to $\bf R \bf$. We assumed the promoter and intitial position both to be at the origin: $z_0 = 0$. Observe that the bulk density $\tilde{c}$ appears as a function of $q$ only in the equation for $\tilde{\xi}$. In order to solve for $\tilde{\xi}$, we first solve the second equation for $\tilde{c}(q,\bf k \rm, s)$, 
\begin{equation}
 \tilde{c}(q,\bf k \rm, s) = \frac{k_d \tilde{\xi}(q,s) - k_a\tilde{c}(q,s)}{s + D_3(q^2 + k^2)} J_{0}(k \sigma ),
\end{equation}
and Fourier back-transform both sides in ${\bf k}$
\begin{eqnarray}
 \elabel{FourierBackC}
 \tilde{c}(q, s) & = & \int \frac{d^2 k}{(2 \pi)^2} \tilde{c}(q,\bf k \rm, s) \\
 & = & \frac{k_d \tilde{\xi}(q,s) - k_a\tilde{c}(q,s)}{2 \pi D_3} \, K_{0}\left( \sigma \sqrt{q^2 + \frac{s}{D_3}} \right).
\end{eqnarray}
Solving the above for $\tilde{c}(q,s)$, and substituting it into \eref{FDE_dna}, we can solve for $\tilde{\xi}(q,s)$. Again, back-transforming this equation in $q$ at the promoter site, we find
\begin{equation}
 \tilde{\xi}(z_0, s) = \int \frac{d\,q}{2 \pi} \frac{1-k_+ \tilde{\xi}(z_0, s)}{s + D_1 q^2+k_d F^{-1}(q,s)}
 \elabel{FDE_dna2}
\end{equation}
where
\begin{equation}
F(q,s) = 1 + \frac{k_a}{2 \pi D_3} \, K_{0}\left( \sigma \sqrt{q^2 + \frac{s}{D_3}} \right).
\end{equation}
Finally, we can solve \eref{FDE_dna2} for $\tilde{\xi}(z_0, s)$ to obtain the probability density at the promoter site in Laplace space. In the limit $s \to 0$, our expression becomes
\begin{equation}
 \lim_{s \to 0} \tilde{\xi}(z_0, s) = \frac{I(\alpha,\beta)}{\pi D_1 / \sigma + k_+ \, I(\alpha,\beta)}
\end{equation}
where
\begin{gather}
 I(\alpha,\beta) = \int_0^{\infty} \frac{dt}{t^2 + \beta[1 + \alpha \, K_0(t)]^{-1}} \\
 \alpha = \frac{k_a}{2 \pi D_3} \elabel{EqAlpha} \\
 \beta = \frac{\sigma^2 k_d}{D_1} \elabel{EqBeta}.
\end{gather}
The $s \to 0$ limit in $k_+ \, \tilde{\xi}$ gives the total flux which flowed into the sink, and this relates to the long time limit of the survival probability
\begin{equation}
  \lim_{t \to \infty}S(t|\sigma) = 1 - k_+ \, \int_0^{\infty} \xi(z_0, t) dt = 1 - k_+ \lim_{s \to 0} \tilde{\xi}(z_0, s) = \frac{\frac{\pi D_1}{\sigma I(\alpha,\beta)}}{\frac{\pi D_1}{\sigma I(\alpha,\beta)} + k_+}.
\end{equation}
The diffusion limited rate constant is
\begin{equation}
 k_D = \frac{\pi D_1}{\sigma I(\alpha,\beta)}.
 \elabel{kDFullJ}
\end{equation}
Note that in the limit of $k_a\to0$ or $D_3\to\infty$, $k_D$ is equal to the well mixed result.

Again, the noise in the promoter state is
\begin{equation}
 \elabel{NoiseProFull}
 \left( \frac{\delta n}{\bar{n}} \right)^2 = 2 \bar{n}(1-\bar{n})\left[\left( \frac{\sigma I(\alpha,\beta)}{\bar{n} \pi D_1 T \bar{\xi}} \right)+\frac{1-\bar{n}}{k_- T \bar{n}^2} \right],
\end{equation}
and the noise in the 'equivalent concentration' of the cytoplasm $c$
\begin{eqnarray}
 \elabel{NoiseCFull} 
 \left( \frac{\delta c}{\bar{c}} \right)^2 & = & \frac{2}{\bar{n}(1-\bar{n})} \left( \frac{k_d}{k_a} \, \frac{\bar{n} \sigma I(\alpha,\beta)}{\pi D_1 T \bar{c}} + \frac{1-\bar{n}}{k_- T}  \right) \\
 \elabel{NoiseCFull2}
 & = & \frac{2}{\bar{n}(1-\bar{n})} \left( \frac{\beta I(\alpha,\beta)}{2 \pi^2 \alpha} \, \frac{\bar{n}}{D_3 T \bar{c}} \, \frac{1}{\sigma} + \frac{1-\bar{n}}{k_- T}  \right).
\end{eqnarray}
In the last line we changed the binding and unbinding rates of the DNA with the dimensionless variables $\alpha$ and $\beta$.


\subsection{Comparing noise in a promoter with a receptor}
We will compare the concentration noise level percieved by a promoter (p, \eref{NoiseCFull2}) with that of a receptor (r obtained in \cite{DeRonde2012}). A receptor is modeld as a sphere lying free in the bulk. Deviding both diffusion limited noise levels, we get
\begin{equation}
 \left( \frac{\delta c_p}{\delta c_r} \right)^2 = \frac{1-\bar{n}_r}{1-\bar{n}_p} \, \frac{k_D^r}{k_D^p} \, \frac{k_d}{k_a}
\end{equation}
where $\bar{n}_r$ and $\bar{n}_p$ are average occupancies of the receptor and promoter respectively. We assumed equal bulk concentrations $\bar{c}$ in both systems with and wihout DNA. In FIGX, the above expression is plotted for different values of $k_a$ and $k_d$. We used the same radii and intrinsic dissociation rates for both the promoter and the receptor. For the association rates we let $k_+^p = k_+^r / 4\pi\sigma_r^2$. Remaining parameters were set to sypical values describes under simulation results.

The vally is caused by two effects. If we let $\beta$ decrease 

On the right side of the valley, increasing $k_a$ or decreasing $k_d$ lowers the noise. This is not obvious, since the rate by which a TF can escape from the promoter, $k_D^r$, decreaes. Since, by increasing $k_a$, DNA rebinding increases and by lowering $k_d$, the DNA residence time increases. However, because TF now stay longer on the DNA, $\bar{\xi}$ increases, and the number of independent measurements a promoter can take during it's integration time also increases. On the right side of the valley, this effects wins from the lower $k_D^r$, On the left side of the valley however, $\bar{\xi}$ becomes so high that the promoter stays occupied for most of the time, and it becomes increasingly difficult to 


\subsection{Comparing results with simulation}

\subsubsection{System setup and parameters}
We check our final result, \eref{NoiseProFull}, by comparing it with simulations done using \GFRD. The simulation setup consists of a box with periodic bounday conditions with a size $L=1\mu m$. Inside is a rod, resembling the DNA, which spans the box and contains one promoter site at it's middle. The system is initialized with TF$_0=10$ randomly placed transcription factors in the bulk. TF's can bind non-specifically to the rod, diffuse one dimensionally and either react with the promoter if it is close, or fall off again. Parameters are set according to results in \cite{Elf2007a}. We set the diffusion constant in the bulk, $D_3=3 \mu m^2/s$, and the diffusion constant on the DNA $D_1=0.05 \mu m^2/s$. The decay rate from the DNA, which sets the average DNA residence time, is $k_d=1ms^{-1}$. The association rate with the DNA is set to a high value, $k_a=10\mu m^2/s$, such that the DNA search becomes diffusion limited. The radius of TF and DNA together is $\sigma=4.4nm$. For the dissociaten rate from the promoter we chose $k_-=0.1 1/ms$. We change the association rate $k_+$, from $5\cdot10^{-5} m/s$ upto $2\cdot10^{-2} m/s$, to compare it's effect on the noise with \eref{NoiseProFull}. 

\subsubsection{Simulation results}
We use \GFRD to generate trajectories of the promoter switching between the occupied or unoccupied state. The easiest way to obtain information from it is by taking the power spectrum of these trajectories. \cite{Tkacik2009,VanZon2006}. In \fref{PowerPlotA} we show the power spectrum of a 3000s long lasting trajectory of the promoter state. It can be devidend into three regimes. The low frequency regime, which is the result of slow processes in the system, is well described by a Poissonian switch. This means that the switching of the promoter on these timescales is induced by TF's which have no prior knowledge of the position of the promoter site. The TF which dissociated from the promoter before, had prior knowledge but escaped, leaving an equilibrated system. The very short timescales, are also dominated by Poissonian processes. These are rebindings of TF's which had no time to diffuse, and the rebind time is set by $k_+$ only. The intermediated regime is dominated by particle trajectories which diffused but remained close to the promoter, before rebinding. This is not a Poisson process, and the Poisson theory doesn't match.

\begin{figure}
\includegraphics[clip]{PowerPlot2}
\caption{\flabel{PowerPlotA} Power spectrum of the promoter state switching generated with simulation (black). Here $k_+=0.16mm/s$, and other parameters are set as described ealier. The colored lines give the power spectra of a Poissonian switch (\eref{PoissonPow}) with a correlation time given by (Red line) \erefstwo{Sinfty}{kDFullJ} or (Green line) $\tau_c^*=k_+ \bar{\xi}+k_-$. The overlap between simulation and theory shows that both at long (low frequency) and short (high frequency) timescales, the switching of a promoter is a Poisson process. The reason is, that on long timescales, their are no TF's spatially correlated with the promoter, and all the binding events come from a well mixed bulk. These bindings are all poisson events because the particles have no prior knowledge of the possition of the promoter. On short timescales, a particle rebinds so quickly, that it had no time to diffuse, and the rebind time is based on the intrisic rate $k_+$ only. The inset shows a power spectrum with a much higher association rate $k_+=19 mm/s$. 
}
\end{figure}

We can obtain the correlation time by taking the low frequency limit of the power spectrum
\begin{equation}
 \lim_{\omega\to0} P_n(\omega) = 2 \sigma_n^2 \tau_c.
 \elabel{PowPro}
\end{equation}
Apart from the factor 2, this is equal to the integral over the correlation function \eref{CT1}, and we will compare our simulation results with this expression. We plot the dependence of the noise as a function of the average occupance $\bar{n}$, where we change $\bar{n}$ by varying the intrinsic association rate $k_+$. The result is shown in \fref{NoisePlotFull}. The theory matches simulation very well up to $\bar{n}\sim0.8$. For higher value's of $\bar{n}$, it is harder to measure the plataue value at the low frequency limit of the power spectrum, as shown in the inset of \fref{PowerPlotA}. Furthermore, for higher $k_+$ values, the low concentration approximation (\aref{noise:Seq_approx}) breaks down, and the offset might be causes by interactions with other particles during rebind trajectories.

\begin{figure}
\includegraphics[clip]{NoisePlotFull}
\caption{\flabel{NoisePlotFull} Low frequency limit of the power spectrum (\eref{PowPro}) for a promoter site on the DNA, compared with simulations and the result found in \cite{Tkacik2009}.
}
\end{figure}


\subsection{A comparison with the paper of Tka$\check{\bf c}$ik \cite{Tkacik2009}}
In his paper, Tka$\check{\bf c}$ik derives the noise for the same systems, but by a very different method. He starts with the differential equations governing the fluctuations $\delta n$ in the promoter state, and relates these fluctuations to changes in the free energy of the promoter due to the binding and unbinding of TF's. Fourier transforming the equations, and solving for $\delta n$, these fluctuations are related to the power spectrum by the fluctuation dissipation theorem. We will discuss the differences in the results for the full system only. The difference in the other simpler systems is analogous to those in the full system. Their final result for the noise in the promoter state is (Eq. 68 in \cite{Tkacik2009}) 
\begin{equation}
 \elabel{NoiseProTka}
 \left( \frac{\delta n}{\bar{n}} \right)^2 = 2 \bar{n}(1-\bar{n})\left[ (1 -\bar{n}) \, \frac{\sigma I(\alpha,\beta)}{\bar{n} \pi^2 D_1 T \bar{\xi}} + \frac{1-\bar{n}}{k_- T \bar{n}^2} \right]
\end{equation}
where $I(\alpha,\beta)$ is given by
\begin{equation}
 \elabel{IntProTka}
 I(\alpha,\beta) = \int_0^{\infty} \frac{dt}{t^2 + 2/\pi^{2} \, \beta[2 + \alpha \, {\rm log}(1 + t^{-2})]^{-1}},
\end{equation}
where $\alpha$ and $\beta$ are defined by \erefstwo{EqAlpha}{EqBeta}. The first important difference is the extra factor $(1-\bar{n})$ in the diffusion term. This makes the noise symmetric in $\bar{n}$, which is clearly wrong as shown in \fref{NoisePlotFull}. Only for small values of $\bar{n}$, when $(1-\bar{n})\sim 1$, the result matches the simulation. The noise in the diffusion part of the concentration, calculated via \aref{noise:promtoconc}, has no dependence on $\bar{n}$ anymore, in contrast to our result in \eref{NoiseCFull} which is proportional to $1/(1-\bar{n})$. However, one would expect the noise in concentration measurements to divergence as the promoter stays occupied for most of the time. Interestingly, de Ronde \cite{DeRonde2012} obtained precisely the same discrepancy for a receptor. Clearly, the extra $(1-\bar{n})$ factor in \eref{NoiseProTka} is wrong, and is most likely the result of a linearization done in \cite{Tkacik2009}. 

The second difference is due to an ultraviolet cutoff needed in \cite{Tkacik2009}. This divergence emerges in the Fourier back-transform analogous to \eref{FourierBackC}, and is the result of the DNA being represented as a cylinder with a zero radius. Physically, such a representation is clearly wrong, because the probability of a point particle finding a line is zero. To circumvent this problem, the upper boundary of the integral is cutoff at one over the smallest lengthscale in the system: $\pi/\sigma$. It is however not clear how well throwing away the higher frequencies compensates for the erronomous geometry. In our derivation (\eref{FDE_bulk}), we give the DNA a finite thickness, and the divergence in \eref{FourierBackC} does not occur. 

Tka$\check{\bf c}$ik ends it's paper with a comparison between the concentration noise level of a promoter and a receptor (3D only); Fig. 2 of \cite{Tkacik2009}. 


\subsection{Obtaining the correlation time by renormalizing the reaction rates}
As was shown by van Zon et al. \cite{VanZon2006}, we can describe our spatially resolved model by a well stirred model if we re-normalize the intrinsic binding and unbinding rates of the promoter $k_+$ and $k_-$ respectively. The idea is that the binding events of TF's to the promoter can be divided in to two separate classes. Either the TF comes from the bulk without any prior knowledge of the position of the promoter, or the TF had just before decayed from the promoter and started it's trajectory in contact with it. The renormalized reaction rates (events per unit of time) $k'_{\rm on}$ and $k'_{\rm off}$ for the well stirred model are found by decreasing the intrinsic rates with all the binding events which are due to a spatial correlation between the TF's and the promoter:
\begin{equation}
 k'_{\rm on} = \frac{k_+}{1 + N_{\rm rb}} \quad \text{and} \quad k'_{\rm off} = \frac{k_-}{1 + N_{\rm rb}}.
\end{equation}
Here $N_{\rm rb}$ is the average number of rebindings before a TF escapes given that is starts in contact with the promoter. The off rate $k_{\rm off}$ is obtained by demanding that the average occupancy of the promoter is unchanged by the renormalization. $N_{\rm rb}$ is calculated straightforwardly via
\begin{equation}
 N_{\rm rb} = \sum_{n=1}^{\infty} n \left( P_{\rm rebind} \right)^n P_{\rm escape} = \frac{1-P_{\rm escape}}{P_{\rm escape}},
\end{equation}
where $P_{\rm rebind}$ and $P_{\rm escape}$ are the splitting probabilities for either escaping or rebinding of a TF at contact with the promoter. The probability of a TF escaping is given by the $t \to \infty$ limit of the survival probability of a particle starting at contact 
\begin{equation}
 P_{\rm escape} = \lim_{t \to \infty} S(t|\sigma) = S(\infty|\sigma).
\end{equation}
The average number of bindings of a particle with the promoter before it escapes again is thus
\begin{equation}
 N_b = 1 + N_{\rm rb} = 1 + \frac{1 - P_{\rm escape}}{P_{\rm escape}} = P_{\rm escape}^{-1} = S(\infty|\sigma)^{-1},
\end{equation}
and the renormalized rates are
\begin{equation}
 k'_{\rm on} = k_+ S(\infty|\sigma) \quad \text{and} \quad  k'_{\rm off} = k_- S(\infty|\sigma).
 \elabel{RenRates}
\end{equation}
The above on-rate is the rate at which TF's bind to the promoter in a system where none of the TF's have prior knowledge of the position of the promoter. Put differently, particles are placed at infinity after falling from the promoter. 

The correlation time for this system is
\begin{equation}
 \tau_c = \frac{1}{k'_{\rm on} \bar{c} + k'_{\rm off}} = \frac{1}{ (k_+ \bar{c} + k_- ) S(\infty|\sigma) } = \frac{\tau_c^*}{S(\infty|\sigma)}.
\end{equation}
This expression for the correlation time is equivalent to that found in \eref{CT1} using the method by de Ronde et al. Thus, in the low concentration limit of non-interacting TF's, the methods of van Zon and de Ronde to obtain the correlation time are equivalent.

\subsubsection{Why the methods are the same}
One important difference between the two methods, is that in the method of van Zon, a TF following a rebinding trajectory, will allways find an unoccupied promoter when it returns. Their is no interuption by the other particles from the equilibrium. De Ronde, a priori, does allow this event by writing the survival probability as a product of either binding from the bulk or rebinding. However, by using the low concentration approximation \eref{Seq_approx}, we throw away the possibility of interuptions on short time scales. We used the large time limit in \aref{noise:Seq_approx}, to show the system is by approximation allways in steady state. By the same arguments, we can show that for small times, the possibility of interuptions are neglectable
\begin{eqnarray}
 \lim_{c\to0} \pdiff{}{t} \mathscr{S}_{\rm irr}(t|\sigma) & = & \lim_{c\to0} \mathscr{S}_{\rm irr}(t|{\rm eq}) \left( \pdiff{}{t}S(t|\sigma) \right)+\left( \pdiff{}{t} \mathscr{S}_{\rm irr}(t|{\rm eq}) \right) S(t|\sigma) \nonumber \\
 & = & \lim_{c\to0} \mathscr{S}_{\rm irr}(t|\sigma) \left( \frac{1}{S(t|\sigma)} \pdiff{}{t} S(t|\sigma) - \bar{c} k(t) \right) \\
 & = & \pdiff{}{t} S(t|\sigma).
\end{eqnarray}
With 'small time' we mean the timescale in which rebindings occur.