\section{The fundamental limit on noise in transcriptional regulation}

\subsection{Introduction}
The time a promoter is occupied by a transcription factor varies strong over time. We want to find out how this variance depends on the mechanisms by which a TF finds it's promoter site. The variance is defined as follows. We measure the time a promoter site is occupied over a time $T$, where $T$ is much larger than the correlation time $\tau_c$. If we repeat this measurement many times, the variance in our collection of measurements is given by (\aref{noise:corr_time})
\begin{equation}
 \left( \delta n \right)^2 \equiv \sigma_{\rm n, T}^2 = \frac{2 \sigma_{\rm n}^2 \,\tau_{\rm c}}{T}
 \elabel{std_devT}
\end{equation}
where $\sigma_n^2$ is the variance of an instantaneous measurement. The promoter is a binomial switch because it has only two states, and for such a process $\sigma^2_n = \bar{n}(1-\bar{n})$, with $\bar{n}$ the average occupancy. In conclusion, finding the correlation time is paramount to finding the noise.

We will first derive the correlation function, from which we can obtain the correlation time. With an expression for the correlation time in hand, we obtain general relations for the noise in the promoter occupancy, concentration fluctuations around the promoter and the lower limit in the noise set by the diffusive arrival of TF's. We continue with obtaining specific expressions for various geometries. Starting with a system where particles can only move on the DNA (1D only), and expanding to DNA in a well mixed solution we end with the full system where TF's can perform multiple excursions allong the DNA and in the cytoplasm before binding to the promoter.


\subsection{Methods}

\subsubsection{The correlation function}
Here we derive the auto-correlation function in Laplace space for a switching process, as was done by de Ronde et al.\cite{DeRonde2012}. Let's start with writing down the general expression for the correlation function of a switch
\begin{eqnarray}
 C_n(\tau) & \equiv & \left\langle (n(\tau) - \bar{n})(n(0) - \bar{n}) \right\rangle \\
  	& = & \bar{n} \left( p_{*|*}(\tau)-\bar{n} \right).
 \elabel{corr_func}
\end{eqnarray}
The function $n(\tau)$ depicts the promoter state and is either one when the promoter is occupied and zero when it is free. In the first line we used that, for a stationary system, we can shift the time such that only one function depends on time. In the second line we introduced the conditional probability that given the promoter was bound at $t=0$, what is the probability that it is bound at a later time $\tau$. This conditional probability is equal to
\begin{equation}
 p_{*|*} = 1 - \mathscr{S}_{\rm rev}(\tau|*)
\end{equation}
where $\mathscr{S}_{\rm rev}(\tau|*)$ is the probability that the promoter is free at time $\tau$, given that it was bound initially. The promoter can undergo multiple rounds of binding and unbinding during the time $\tau$. We can describe this reversible process in term of an irreversible one via the convolution \cite{Agmon1990}
\begin{equation}
 \mathscr{S}_{\rm rev}(t|*) = k_- \int_0^t [1-\mathscr{S}_{\rm rev}(t'|*)]\mathscr{S}_{\rm rad}(t-t'|z_0)dt'.
 \elabel{rev_conv}
\end{equation}
The first factor under the integral gives the probability that the promoter is occupied at time $t'$. Then it decays for the last time with a rate $k_-$, and remains unoccupied until $t$ with a probability $\mathscr{S}_{\rm rad}(t-t'|z_0)$. After decay, a TF is placed at contact (denoted by $z_0$ in $\mathscr{S}_{\rm rad}$) with the promoter. Integrating over all intermediate times $t'$, gives us the probability that a promoter is unoccupied at time $t$. 

For the irreversible survival probability with one TF at contact with the promoter we propose the form
\begin{equation}
 \mathscr{S}_{\rm rad}(t|z_0) = \mathscr{S}_{\rm rad}(t|{\rm eq}) S(t|z_0)
 \elabel{Def_Ssi}
\end{equation}
where $\mathscr{S}_{\rm rad}(t|{\rm eq})$ is the survival probability of a free promoter site surrounded by an equilibrated solution of TF's. $S_{\rm rad}(t|z_0)$ is the survival probability with just one TF placed on the promoter site. This form implies that after a TF has dissociated from the promoter, all the other TF's are equilibrated. This is a good approximation when the average decay time $1/k_-$ is long enough for another particle decayed earlier, to equilibrate. 

$\mathscr{S}_{\rm rad}(t|{\rm eq})$ can be found by solving the differential equation
\begin{equation}
 \pdiff{\mathscr{S}_{\rm rad}(t|{\rm eq})}{t} = - \bar{\xi} \, k_{\rm rad}(t) \, \mathscr{S}_{\rm rad}(t|{\rm eq}).
 \elabel{DE_Seq}
\end{equation}
which states that the rate at which TF's bind to the promoter is equal to the rate $k_{{\rm rad}}(t)$ at with they would bind to the promoter if it is never occupied, times the probability that the promoter is still unoccupied. $\bar{\xi}$ is the average concentration of TF's on the DNA. Solving the equation yields
\begin{equation}
 \mathscr{S}_{\rm rad}(t|{\rm eq}) = e^{-\bar{\xi}\int_0^t k_{{\rm rad}}(t') dt'},
 \elabel{Seq}
\end{equation}
and $k_{{\rm rad}}(t)$ is, using the backward Smoluchowski equation (\aref{noise:backward_eqn}), equal to 
\begin{equation}
 k_{\rm rad}(t) = k_+ \, S_{\rm rad}(t|z_0).
 \elabel{krad}
\end{equation}
Before diving into the Laplace domain, we give a relation which will prove to be very useful. Namely, from \erefstwo{DE_Seq}{krad}, it is clear that
\begin{eqnarray}
 \pdiff{\mathscr{S}_{\rm rad}(t|{\rm eq})}{t} & = & -\bar{\xi} k_+ \, S_{\rm rad}(t|z_0) \, \mathscr{S}_{\rm rad}(t|{\rm eq}) \\
 & = & -\bar{\xi} k_+ \, \mathscr{S}_{\rm rad}(t|z_0).
 \elabel{SeqSsi}
\end{eqnarray}

We start by Laplace transforming \eref{rev_conv} and solve it for $\mathscr{S}_{\rm rev}$. By using the Laplace transformed \erefstwo{corr_func}{SeqSsi}, we can express the correlation function as a function of $\hat{\mathscr{S}}_{\rm rad}(s|{\rm eq})$ only
\begin{equation}
 \hat{C_n}(s) = \sigma_n^2 \, \frac{\bar{n} \hat{\mathscr{S}}_{\rm rad}(s|{\rm eq})}{1-(1-\bar{n})s\hat{\mathscr{S}}_{\rm rad}(s|{\rm eq})}.
 \elabel{LapCorr}
\end{equation}
Since we can not directly Laplace transform the exponential in \eref{Seq}, we argue in \aref{noise:Seq_approx} that, in the limit of low concentrations $\bar{\xi}$, it can be approximated as
\begin{equation}
 \lim_{c\to0} \hat{\mathscr{S}}_{\rm rad}(s|{\rm eq}) = \frac{1}{s}  \, \frac{1}{ 1 + \bar{\xi} \hat{k}_{\rm rad}(s)}.
 \elabel{Seq_approx}
\end{equation}
Substituting this approximation into \eref{LapCorr}, we obtain, after simplifying
\begin{equation}
 \lim_{c \to 0} \hat{C_n}(s) = \frac{\bar{n} \sigma_n^2}{\bar{n} s + k_+ \bar{\xi} \, s \hat{S}_{\rm rad}(s|z_0)}.
 \elabel{Corr_approx}
\end{equation}
The correlation time of the process is found by taking the $s \to 0$ limit
\begin{eqnarray}
 \lim_{s \to 0} \hat{C_n}(s) = \int_0^\infty C_n(t) dt = \sigma_n^2 \, \tau_c.
\end{eqnarray}
Using the approximation in \eref{Corr_approx}, the general expression for the correlation time becomes (\aref{noise:limits_Laplace})
\begin{equation}
 \lim_{c \to 0, s \to 0} \hat{C_n}(s) = \frac{\bar{n} \sigma_n^2}{k_+ \bar{\xi} \, S_{\rm rad}(\infty|z_0)} = \frac{\sigma_n^2 \tau_c^*}{S_{\rm rad}(\infty|z_0)},
 \elabel{CT1}
\end{equation}
where $\tau_c^*$ is the correlation time for the fast process, $\tau_c^* = k_+ \hat{\xi} + k_-$. Note that in geometries for which the particle always returns to the starting point, such as 1D and 2D diffusion problems, $S_{\rm rad}(t\to\infty|z_0)\to0$, and the correlation time $\tau_c$ diverges. Because a particle can't escape, it will allways be correlated with it's starting point. For these processes, we adopt a different defenition for the variance, as is defined in \eref{EqProNoiseII}.

To summarize, given the large time limit of $S(t|z_0)$, we have, in the low concentration limit, the correlation time.


\subsubsection{Defining noise}
We define the noise in the receptor state, combining \erefstwo{std_devT}{CT1}, as
\begin{equation}
 \left( \frac{\delta n}{\bar{n}} \right)^2 = 2 \, \frac{\sigma_n^2}{\bar{n}^2} \, \frac{\tau_c}{T} = 2 \, \frac{\sigma_n^2}{\bar{n}^2} \, \frac{\tau_c^*}{S_{\rm rad}(\infty|z_0)} \, \frac{1}{T}.
 \elabel{EqProNoise}
\end{equation}
Using Eq. 2.27 in \cite{Agmon1990}, the escape probability and the correlation time $\tau_c$ can be written as
\begin{equation}
 S_{\rm rad}(\infty|z_0) = \frac{k_D}{k_+ + k_D} \quad {\rm and} \quad \tau_c = \frac{k_+ + k_D}{(k_+ \bar{\xi} + k_-) k_D},
 \elabel{Sinfty}
\end{equation}
where $k_+$ and $k_-$ are the intrinsic association and dissociation rates of the promoter site respectively, and $k_D$ the rate at which TF's arrive at an 'absorbing' promoter ($k_+ \to \infty$, a TF allways reacts when it `hit's` the promoter site.) in steady-state. We note that for systems where $S(t\to\infty|z_0)\to0$, also $k_D\to0$, because particles 'from inifity' can't arrive fast enough to replenish the depletion due to the sink. 

Combining our relations in \eref{Sinfty} with \eref{EqProNoise}, \eref{EqxiNoise} and \eref{EqcNoise}, we arrive at general relations for the noise in the receptor
\begin{eqnarray}
 \elabel{NoisePro}
 \left( \frac{\delta n}{\bar{n}} \right)^2 & = & 2 \bar{n}(1-\bar{n})\left[\left( \frac{1}{\bar{n} k_D T \bar{\xi}} \right)+\frac{1-\bar{n}}{k_- T \bar{n}^2} \right] \\
 & = & \frac{2 \sigma_n^2}{T \bar{\xi} \bar{n}} \, \frac{1}{k_{\rm on}},
\end{eqnarray}
the noise in the concentration on the DNA, derived from the promoter state
\begin{eqnarray}
 \elabel{NoiseXi}
 \left( \frac{\delta \xi}{\bar{\xi}} \right)^2 & = & \frac{2}{\bar{n}(1-\bar{n})} \left( \frac{\bar{n}}{k_D T \bar{\xi}} + \frac{1-\bar{n}}{k_- T}  \right) \\
  & = & \frac{1}{T \bar{\xi} (1 - \bar{n})} \, \frac{1}{k_{\rm on}},
\end{eqnarray}
and the noise in the 'equivalent concentration' of the cytoplasm $c$
\begin{eqnarray}
 \elabel{NoiseC} 
 \left( \frac{\delta c}{\bar{c}} \right)^2 & = & \frac{2}{\bar{n}(1-\bar{n})} \left( \frac{k_d}{k_a} \, \frac{\bar{n}}{k_D T \bar{c}} + \frac{1-\bar{n}}{k_- T}  \right) \\ \elabel{NoiseC2} 
 & = & \frac{1}{T \bar{c} (1 - \bar{n})} \, \frac{k_d}{k_a} \, \frac{1}{k_{\rm on}}
\end{eqnarray}
We can find the lower bound for the noise, or the noise due to diffusion only, by removing all the terms having a $k_-$ in \erefsrange{NoisePro}{NoiseC}. Now, the 'radiation' on-rate $k_{\rm on}$ changes in the diffusion limited on-rate $k_D$ in the second line of all the expressions above. 

The final expression in \eref{NoiseC2} has an intuitive interpretation: measuring a time $T$, the number of new TF's arriving from the bulk and reacting with the promoter is $\bar{c}\,k_{\rm on}\,T$, on average.  Because the promoter can't 'register' TF's when it is occupied, the maximum number of TF's available is decreased by the fraction of time the promoter is not occupied. Thus, the noise is one over the number of TF's the promoter reacts with during a time $T$.

For some processes, such as diffusion along a line or on a plane, the correlation time diverges to infinity, and the promoter won't be able to make true independent measurements. As a result, the noise perceived by a promoter is infinite given the definition in \eref{EqProNoise}. Therefore, for these kind of processes, we adopt a different definition
\begin{equation}
  \left( \frac{\delta n}{\bar{n}} \right)^2 = \left. \frac{1}{\bar{n}^2} \lim_{t \to \infty} C_n(t)  \right|_{t = T}.
  \elabel{EqProNoiseII}
\end{equation}
Here, the noise decays as a function of the integration time $T$, as the long time behavior of the correlation function.


\subsection{Noise in different environments}

\subsubsection{1D diffusion only}
We start by calculating the noise in a system where TF's diffuse along the DNA only: no excursions in the cytoplasm are allowed. It is well known \cite{Redner2001} that for a 1D system, particles always return to their starting position. As a result, TF's are unable to 'escape' from binding to the promoter and the survival probability of the single particle problem tends to zero. Therefore, we use \eref{EqProNoiseII} for defining the noise, and we start by deriving the large time limit of the correlation function.

The survival probability in Laplace space for this system is (\eref{LSOL_NB})
\begin{equation}
 \hat{S}_{\rm rad}(s|z_0) = \frac{1}{s}\left( \frac{\sqrt{4 D_1 s}}{\sqrt{4 D_1 s} + k_+} \right),
 \elabel{LSx0}
\end{equation}
where $D_1$ is the diffusion constant in 1D and $k_+$ is the association rate of the promoter. Plugging this into the low concentration approximation for the correlation function \eref{Corr_approx}
\begin{equation}
  \hat{C_n}(s) = \sigma_n^2 \, \frac{\bar{n} \, \left( k_+ + \sqrt{4 D_1 s} \right)}{k_+ \bar{n} s + \bar{\xi} k_+ \sqrt{4 D_1 s} + \bar{n} s \sqrt{4 D_1 s}}.
  \elabel{LapCorr1D}
\end{equation}
It is clear that in the limit $s \to 0$, the correlation function diverges to $+\infty$ such that $\tau_c$ is infinite. To lowest order, the expansion in $s$ reads
\begin{equation}
 \lim_{s \to 0} \hat{C_n}(s) = \frac{\bar{n}^2(1-\bar{n})}{\bar{\xi} \sqrt{4 D_1 s}}.
\end{equation}
Transforming this back to the time domain, we find the limit for large $t$
\begin{equation}
 \lim_{t \to \infty} \hat{C_n}(t) = \frac{\bar{n}^2(1-\bar{n})}{\bar{\xi} \sqrt{4 \pi D_1 t}}.
\end{equation}
The correlation function of the concentration in the large time limit is
\begin{equation}
 \lim_{t \to \infty} \hat{C_{\xi}}(t) = \lim_{t \to \infty} \left| \frac{\partial n}{\partial \xi} \right|^{-2} \hat{C_n}(t) = \frac{\bar{\xi}}{(1 - \bar{n}) \sqrt{4 \pi D_1 t}}.
\end{equation}
Using the alternative definition of the noise in \eref{EqProNoiseII}, we find
\begin{equation}
 \left( \frac{\delta n}{\bar{n}} \right)^2 =  \frac{(1-\bar{n})}{\bar{\xi} \sqrt{4 \pi D_1 T}} \quad \text{and} \quad \left( \frac{\delta \xi}{\bar{\xi}} \right)^2 = \frac{1}{ \bar{\xi} (1 - \bar{n}) \sqrt{4 \pi D_1 T}}. 
 \elabel{Noise1D}
\end{equation}
We see that the noise decreases only with the square root of the measurement time. This is due to the fact that a TF never escapes from the promoter, and it is therefore impossible for the promoter to take truly independent measurements of the TF concentration. 


\subsubsection{1D diffusion with exchange to a perfectly mixed bulk}
Here we imagine a system where TF's diffuse along an infinite strand of DNA and dissociate from it with a uniform rate $k_d$. When they dissociate, they are placed at a random position in the bulk. The survival probability for a TF initially at the promoter site is given by \eref{LSx0}, where, due to the uniform dissociation from the DNA, the Laplace variable $s$ is shifted by $k_d$: $\sqrt{s/D} \to \sqrt{(s+k_d)/D}$. The required long time limit of the survival probability is now larger than zero
\begin{equation}
 \lim_{s\to0} s \, \hat{S}_{\rm rad}(s|z_0) = S_{\rm rad}(\infty|z_0) = \frac{\sqrt{4 D_1 k_d}}{\sqrt{4 D_1 k_d} + k_+}.
\end{equation}
Comparing this with \eref{Sinfty}, the diffusion limited on-rate for this system is
\begin{equation}
k_D = \sqrt{4 D_1 k_d}.
\end{equation}
Using the noise \erefsrange{NoisePro}{NoiseC}, we find that the noise in the promoter state
\begin{equation}
 \elabel{NoiseProWM}
 \left( \frac{\delta n}{\bar{n}} \right)^2 = 2 \bar{n}(1-\bar{n})\left[\left( \frac{1}{\bar{n} \sqrt{4 D_1 k_d} T \bar{\xi}} \right)+\frac{1-\bar{n}}{k_- T \bar{n}^2} \right],
\end{equation}
and the noise in the 'equivalent concentration' of the cytoplasm $c$
\begin{equation}
 \elabel{NoiseCWM} 
 \left( \frac{\delta c}{\bar{c}} \right)^2 = \frac{2}{\bar{n}(1-\bar{n})} \left( \frac{1}{k_a} \, \frac{\bar{n}}{\sqrt{4 D_1 k_d^{-1}} T \bar{c}} + \frac{1-\bar{n}}{k_- T}  \right).
\end{equation}


\subsubsection{1D diffusion with reversible binding to the DNA}
This time, we want to evaluate the noise in the occupancy of a promoter site where particles do not escape after falling from the DNA, but are placed at contact with it. This way, a TF starting at the promoter site can switch several times between sliding along the DNA and making excursions in the cytoplasm before either rebinding or escaping, as is sketched in \fref{promoter_schematic}. 

To calculate the variance in the promoter occupancy, we need the corrrelation time. Remember that the correlation time is inversely proportional to the probability by which a single TF, starting on the promoter site, escapes: $\tau_c = \tau_c^* / S_{\rm rad}(\infty|z_0)$. To find this long time limit of the survival probability, we write down the full system of diffusion equations governing the behavior of a single TF starting on the promoter site. It turns out we can solve these equations, and find an exact solution for $S_{\rm rad}(\infty|z_0)$. Some of the mathematics in our derivation is inspired by the paper of Tka$\check{\rm c}$ik \cite{Tkacik2009}. 

\begin{figure}[ht]
\centering
\includegraphics[scale=.5]{promoter_schematic}
\caption{\flabel{promoter_schematic} Schematic representation of the system. We want to find the survival probability of a single TF starting on the promoter site located at $z_0$. The TF slides along the DNA with a small diffusion constant $D_1$ and can irreversibly bind to the promoter with a rate $k_+$. Or it falls from the DNA with a rate $k_d$, and is placed in the bulk a distance $\sigma$ from the z-axis. Here it starts diffusing around in the cytoplasm with a much larger diffusion constant $D_3$. The bulk diffusion is essentially a 2D problem and the TF will eventually rebind with the DNA with an intrinsic rate $k_a$. Note that we want to derive the irreversible survival probability, so the TF can not unbind once it is bound to the promoter: $k_- = 0$.}
\end{figure}

We start with writing down the differential equations describing the full system
\setlength{\jot}{12pt}
\begin{equation}
\elabel{DE_full}
\begin{split}
\pdiff{P_{1}(z,t|z_0)}{t} & = D_1 \frac{\partial^2 P_{1}(z,t|z_0)}{\partial^2 z} - k_+ P_{1}(z,t|z_0) \delta(z - z_0) - k_d P_{1}(z,t|z_0) \\
 & \quad  + k_a P_{3}(z, |{\bf r}| = \sigma, t|z_0, r_0) \\
\pdiff{P_{3}(z,{\bf r},t|z_0,r_0)}{t} & = D_3 \nabla^2 P_{3}(z, {\bf r}, t|z_0,r_0) \\
 & \quad - \Big[ k_a P_{3}(z, {\bf r}, t|z_0,r_0) - k_d P_{1}(z,t|z_0) \Big] \frac{\delta(|{\bf r}| - \sigma)}{2 \pi \sigma}.
\end{split}
\end{equation}
Here $P_1(z, t|z_0)$ is the Green's function describing the 1D sliding of the TF along the DNA, starting on the sink positioned at $z_0$. We model the DNA as an infinite long rod along the z-axis. The probability exchange between the DNA and bulk happens at a distance $\sigma$ from the z-axis, dictated by the delta function. Excursions in the cytoplasm are described by $P_3(z, {\bf r}, t|z_0,r_0)$, where $r_0=0$, stating that the particle starts on the DNA. First we Laplace transform with respect to time
\begin{eqnarray*}
 s \, \hat{P}_{1}(z) - \delta(z - z_0) & = & D_1 \frac{\partial^2\hat{P}_1(z)}{\partial^2 z}  - k_+ \hat{P}_1(z) \delta(z - z_0) - k_d \hat{P}_1(z) + k_a \hat{P}_3(z, |{\bf r}| = \sigma) \\
 s \, \hat{P}_3(z,{\mathbf r}) & = & D_3 \nabla^2 \hat{P}_3(z, {\bf r}) - \left[ k_a \hat{P}_3(z,{\bf r}) - k_d \hat{P}_1(z) \right] \frac{\delta(|{\bf r}| - \sigma)}{2 \pi \sigma},
\end{eqnarray*}
where we explicitly included the initial condition of one particle placed in contact with the promoter site on the DNA. We left away the Laplace variable $s$ and the initial conditions in the function arguments. We continue by Fourier transforming with respect to space
\begin{eqnarray}
 s \, \tilde{P}_1(q) - 1 & = & - D_1 q^2 \tilde{P}_1(q) - k_+ \tilde{P}_1(z_0) - k_d \tilde{P}_1(q) + k_a \tilde{P}_3(q) \elabel{FDE_dna} \\
 s \, \tilde{P}_3(q,{\bf k}) & = & - D_3 (q^2 + k^2) \tilde{P}_3(q, {\bf k}) - \left[ k_a \tilde{P}_3(q) - k_d \tilde{P}_1(q) \right] J_{0}(k \sigma). 
 \elabel{FDE_bulk}
\end{eqnarray}
Here $q$ is the spatial Fourier variable conjugate to $z$, and ${\bf k}$ is conjugate to ${\bf r}$. $J_0(k\sigma)$ is the zeroth order bessel function of the first kind, resulting from the Fourier integral as is shown in \aref{noise:delta_int}. We take both the promoter and initial position to be at the origin: $z_0 = 0$. Observe that the cytoplasmic density $\tilde{P}_3$ appears as a function of $q$ only in the equation for $\tilde{P}_1$. In order to solve for $\tilde{P}_1$, we first solve the second equation for $\tilde{P}_3(q,{\bf k})$, 
\begin{equation}
 \tilde{P}_3(q,{\bf k}) = \frac{k_d \tilde{P}_1(q) - k_a\tilde{P}_3(q)}{s + D_3(q^2 + k^2)} J_{0}(k \sigma ).
\end{equation}
Fourier back-transforming both sides of the equation in ${\bf k}$, at $|{\bf r}|=\sigma$,
\begin{eqnarray}
 \elabel{FourierBackC}
 \tilde{P}_3(q) & = & \int \frac{d^2 k}{(2 \pi)^2} {\rm exp}(-i\,{\bf k}\cdot{\bf r}) \delta(|{\bf r}|-\sigma) \tilde{P}_3(q,{\bf k}) \\
 & = & \frac{k_d \tilde{P}_1(q) - k_a\tilde{P}_3(q)}{2 \pi D_3} \, I_{0}\left( \sigma \sqrt{q^2 + \frac{s}{D_3}} \right) K_{0}\left( \sigma \sqrt{q^2 + \frac{s}{D_3}} \right),
\end{eqnarray}
where $I_0$ and $K_0$ are the zeroth order modified bessel functions of the first and second kind respectivily. Solving the above for $\tilde{P}_3(q)$, and substituting the result into \eref{FDE_dna}, we obtain the solution for $\tilde{P}_1(q)$. Again, back-transforming this equation in $q$, we find
\begin{equation}
 \tilde{P}_1(z_0,s|z_0) = \int \frac{d\,q}{2 \pi} \frac{1-k_+ \tilde{P}_1(z_0,s|z_0)}{s + D_1 q^2+k_d F^{-1}(q,s)}
 \elabel{FDE_dna2}
\end{equation}
where
\begin{equation}
F(q,s) = 1 + \frac{k_a}{2 \pi D_3} \, I_{0}\left( \sigma \sqrt{q^2 + \frac{s}{D_3}} \right) K_{0}\left( \sigma \sqrt{q^2 + \frac{s}{D_3}} \right).
\end{equation}
Finally, we can solve \eref{FDE_dna2} for $\tilde{P}_1(z_0, s|z_0)$ to obtain the probability density at the promoter site in Laplace space. In the limit $s \to 0$, our expression becomes
\begin{equation}
 \lim_{s \to 0} \tilde{P}_1(z_0, s) = \frac{I(\alpha,\beta)}{\pi D_1 / \sigma + k_+ \, I(\alpha,\beta)}
\end{equation}
where
\begin{gather}
 I(\alpha,\beta) = \int_0^{\infty} \frac{dt}{t^2 + \beta[1 + \alpha \, I_0(t) K_0(t)]^{-1}} \\
 \alpha = \frac{k_a}{2 \pi D_3} \elabel{EqAlpha} \\
 \beta = \frac{\sigma^2 k_d}{D_1} \elabel{EqBeta}.
\end{gather}
The flux into the promoter at any given time is $k_+\,P_1(z_0,t|z_0)$, and the total flux which leaks away through the promoter is equal to the $s \to 0$ limit in $k_+ \,\hat{P}_1(z_0,s|z_0)$. This relates to the long time limit of the survival probability via
\begin{gather}
  \lim_{t \to \infty}S_{\rm rad}(t|z_0) = 1 - k_+ \, \int_0^{\infty} P_1(z_0, t|z_0) dt = 1 - k_+ \lim_{s \to 0} \hat{P}_1(z_0, s|z_0) \\
= \frac{\frac{\pi D_1}{\sigma I(\alpha,\beta)}}{\frac{\pi D_1}{\sigma I(\alpha,\beta)} + k_+}.
\end{gather}
Comparing with \eref{Sinfty}, the diffusion limited rate constant is
\begin{equation}
 k_D = \frac{\pi D_1}{\sigma I(\alpha,\beta)}.
 \elabel{kDFullJ}
\end{equation}
Note that in the limit of $k_a\to0$ or $D_3\to\infty$, $k_D$ is equal to the well mixed result.

Plugging in our result for $k_D$ into \erefstwo{NoisePro}{NoiseC}, the noise in the promoter state is
\begin{equation}
 \elabel{NoiseProFull}
 \left( \frac{\delta n}{\bar{n}} \right)^2 = 2 \bar{n}(1-\bar{n})\left[\left( \frac{\sigma I(\alpha,\beta)}{\bar{n} \pi D_1 T \bar{\xi}} \right)+\frac{1-\bar{n}}{k_- T \bar{n}^2} \right],
\end{equation}
and the noise in the 'equivalent concentration' of the cytoplasm $c$
\begin{eqnarray}
 \elabel{NoiseCFull} 
 \left( \frac{\delta c}{\bar{c}} \right)^2 & = & \frac{2}{\bar{n}(1-\bar{n})} \left( \frac{k_d}{k_a} \, \frac{\bar{n} \sigma I(\alpha,\beta)}{\pi D_1 T \bar{c}} + \frac{1-\bar{n}}{k_- T}  \right) \\
 \elabel{NoiseCFull2}
 & = & \frac{2}{\bar{n}(1-\bar{n})} \left( \frac{\beta I(\alpha,\beta)}{2 \pi^2 \alpha} \, \frac{\bar{n}}{D_3 T \bar{c}} \, \frac{1}{\sigma} + \frac{1-\bar{n}}{k_- T}  \right).
\end{eqnarray}
In the last line we changed the binding and unbinding rates of the DNA with the dimensionless variables $\alpha$ and $\beta$.


\subsection{Comparing noise in a promoter with a receptor}
To gain better insight into the noise in the TF concentration perceived by a promoter, \eref{NoiseCFull}, we will compare it with the noise in a receptor. The noise in a receptor was derived in \cite{DeRonde2012}, and is modeled as a reactive sphere lying free in the bulk. Using labels $r$ for receptor, and $p$ for promoter, we divide the diffusion limited part of both the noise levels to obtain
\begin{eqnarray}
 \left( \frac{\delta c_p / \bar{c}_p}{\delta c_r / \bar{c}_r} \right)^2 & = & \frac{1-\bar{n}_r}{1-\bar{n}_p} \, \frac{k_D^r}{k_D^p} \, \frac{\bar{c}_r}{\bar{c_p}} \frac{k_d}{k_a} \\
 & = & \frac{1-\bar{n}_r}{1-\bar{n}_p} \, \frac{k_D^r}{k_D^p} \, \frac{k_d}{k_a} \left( 1 + \frac{k_a}{k_d L^2}\right),
 \elabel{NoiseCmpA}
\end{eqnarray}
where $\bar{n}_r$ and $\bar{n}_p$ are average occupancies of the receptor and promoter respectively. $k_D$ is the rate at which particles arrive at an absorbing sink or sphere, in steady-state. In the second line we enforce that the total number of transcription factors, TF$_0$, is conserved: $\bar{\xi} \, L + \bar{c}_p \, L^3 = {\rm TF}_0$ and $\bar{c}_r = {\rm TF}_0/L^3$. We are interested in the behavior of the noise when we change the association and dissociation rates of the DNA. The other parameters were set to typical values described in the simulation results section. 

A plot of \eref{NoiseCmpA} for different values of $k_d$ and $k_a$ is shown in \fref{TableauxNoiseCmp} (a). The optimum is caused by two opposing effects setting the noise level. Starting on the right side, decreasing the DNA dissociation-rate (or increasing the association rate) increases the concentration $\bar{\xi}$ of TF's on the DNA, and thereby increasing the number of independent measurements a promoter can make during it's integration time. However, as the promoter becomes occupied for more than half of the time, \fref{TableauxNoiseCmp} (b), it doesn't `measure' the greater part of the incoming particles any more, and the noise starts to rise again. Even when we ignore the effect of promoter occupancy, the noise has a valley at higher residence times (c). Here, the majority of TF's in the cell will occupy the DNA, and $\bar{\xi}$ doesn't increase any more with decreasing $k_d$. At this point, the increasing difficulty of TF's to escape from the promoter starts to dominate (decreasing $k_D$), and again the noise increases. Note that, in reality, the DNA is much longer than the dimension of a cell, and the effect of the saturation of $\bar{\xi}$ dominates at much lower values of $k_d$. Finally, for a system with infinite TF's, and ignoring promoter occupancy, the noise increases monotonically with increasing $k_d$ (d). 

\begin{figure}
\includegraphics[clip]{TableauxNoiseCmp}
\caption{\flabel{TableauxNoiseCmp} Comparing the noise in a promoter with a receptor. In all plots, the DNA dissociation rate, $k_d$, is on the horizontal axis. Different lines are for different values of the DNA association rate $k_a$. A legend with values is given in figure c. {\bf a,b}) Noise perceived by a promoter relative to a receptor. Roughly at the point when the promoter is occupied for half of the time, their is a minimum in the noise. {\bf c}) Removing the effect of promoter occupancy, the noise shows a new valley due to the finite number of TF's in the system. {\bf d}) Removing both the constrain of promoter occupancy and finiteness of TF's, the noise has no minimum.
}
\end{figure}



\subsection{Comparing results with simulation}

\subsubsection{System set-up and parameters}
We check our final result, \eref{NoiseProFull}, by comparing it with simulations done using \GFRD. The simulation set-up consists of a box with periodic boundary conditions with a size $L=1\mu m$. Inside is a rod, resembling the DNA, which spans the box and contains one promoter site at it's middle. The system is initialized with TF$_0=10$ randomly placed transcription factors in the bulk. TF's can bind non-specifically to the rod, diffuse one dimensionally and either react with the promoter if it is close, or fall off again. Parameters are set according to results in \cite{Elf2007a}. We set the diffusion constant in the bulk, $D_3=3 \mu m^2/s$, and the diffusion constant on the DNA $D_1=0.05 \mu m^2/s$. The decay rate from the DNA, which sets the average DNA residence time, is $k_d=1ms^{-1}$. The association rate with the DNA is set to a high value, $k_a=10\mu m^2/s$, such that the DNA search becomes diffusion limited. The radius of TF and DNA together is $\sigma=4.4nm$. For the dissociation rate from the promoter we chose $k_-=1 ms^{-1}$. We change the association rate $k_+$, from $5\cdot10^{-5} m/s$ up-to $2\cdot10^{-2} m/s$, to compare the effect on the noise in the simulation with our analytical solution in \eref{NoiseProFull}. 

\subsubsection{Simulation results}
We use \GFRD\, to generate trajectories of the promoter switching between the occupied or unoccupied state. The easiest way to obtain information from it is by taking the power spectrum of these trajectories \cite{Tkacik2009,VanZon2006}. In \fref{PowerPlotA} we show the power spectrum of a 3000s long lasting trajectory of the promoter state. It can be dividend into three regimes. The low frequency regime, which is the result of slow processes in the system, is well described by a Poissonian switch. This means that the switching of the promoter on these time-scales is induced by TF's which have no prior knowledge of the position of the promoter site. The TF which dissociated from the promoter before, had prior knowledge but escaped, leaving an equilibrated system. The very short time-scales are also dominated by Poissonian processes. These are rebindings of TF's which had no time to diffuse, and the rebind time is set by $k_+$ only. In the intermediated regime, waiting times are dominated by particles which diffused around, but did not escape, before rebinding. Due to it's recent dissociation, the TF has prior knowledge of the promoter position, and the waiting times are non Poissonian.

\begin{figure}
\includegraphics[clip]{PowerPlot2}
\caption{\flabel{PowerPlotA} Power spectrum of the promoter state switching generated with simulation (black). Here $k_+=0.16mm/s$, and other parameters are set as described earlier. The coloured lines give the power spectra of a Poissonian switch defined in \aref{noise:corr_time}, \eref{PoissonPow}. For the red line, the correlation time is given by \eref{Sinfty}, where $k_D$ is definined in \eref{kDFullJ}. The green line, at high frequencies, is a switch with correlation time $(k_+ \bar{\xi}+k_-)^{-1}$. The overlap between simulation and theory shows that both at long (low frequency) and short (high frequency) time-scales, the switching of a promoter is a Poisson process. The reason is, that on long time-scales, their are no TF's spatially correlated with the promoter, and all the binding events come from a well mixed bulk. These bindings are all Poisson events because the particles have no prior knowledge of the position of the promoter. On short time-scales, a particle rebinds so quickly, that it had no time to diffuse, and the rebind time is based on the intrinsic rate $k_+$ only. The inset shows a power spectrum with a much higher association rate $k_+=19 mm/s$. 
}
\end{figure}

To measure the correlation time, we take the low frequency limit of the power spectrum (\aref{noise:corr_time})
\begin{equation}
 \lim_{\omega\to0} P_n(\omega) = 2 \sigma_n^2 \tau_c.
 \elabel{PowPro}
\end{equation}
Apart from the factor 2, this is equal to the integral over the correlation function \eref{CT1}, and we will compare our simulation results with this expression. We plot the dependence of the noise as a function of the average occupancy $\bar{n}$, where we change $\bar{n}$ by varying the intrinsic association rate $k_+$. The result is shown in \fref{NoisePlotFull}. The theory matches simulation very well up to $\bar{n}\sim0.8$. For higher value's of $\bar{n}$, it is harder to measure the plateau value at the low frequency limit of the power spectrum, as shown in the inset of \fref{PowerPlotA}. Furthermore, for higher $k_+$ values, the low concentration approximation (\aref{noise:Seq_approx}) breaks down, and the offset might be caused by interactions with other particles during rebinding trajectories.

\begin{figure}
\centering
\includegraphics[scale=.75]{NoisePlotFull}
\caption{\flabel{NoisePlotFull} Low frequency limit of the power spectrum (\eref{PowPro}) for a promoter site on the DNA, compared with simulations. The grey lines shows the result found in \cite{Tkacik2009}, which, in contrast to our result, is symmetric around $n=.5$.}
\end{figure}



\subsection{Comparing with the paper of Tka$\check{\bf c}$ik}
In his paper, Tka$\check{\rm c}$ik derives the noise for the same systems as we did, but by a very different method. He starts with the differential equations governing the fluctuations $\delta n$ in the promoter state, and relates these fluctuations to changes in the free energy of the promoter due to the binding and unbinding of TF's. Fourier transforming the equations, and solving for $\delta n$, these fluctuations are related to the power spectrum by the fluctuation-dissipation theorem. We will discuss the differences in the results for the full system only. The difference in the other simpler systems is analogous to those in the full system. 

The final result for the noise in the promoter state is (Eq. 68 in \cite{Tkacik2009}) 
\begin{equation}
 \elabel{NoiseProTka}
 \left( \frac{\delta n}{\bar{n}} \right)^2 = 2 \bar{n}(1-\bar{n})\left[ (1 -\bar{n}) \, \frac{\sigma I(\alpha,\beta)}{\bar{n} \pi^2 D_1 T \bar{\xi}} + \frac{1-\bar{n}}{k_- T \bar{n}^2} \right]
\end{equation}
where $I(\alpha,\beta)$ is given by
\begin{equation}
 \elabel{IntProTka}
 I(\alpha,\beta) = \int_0^{\infty} \frac{dt}{t^2 + 2/\pi^{2} \, \beta[2 + \alpha \, {\rm log}(1 + t^{-2})]^{-1}},
\end{equation}
and $\alpha$ and $\beta$ are defined in \erefstwo{EqAlpha}{EqBeta}. 

The first important difference is the extra factor $(1-\bar{n})$ in the diffusion term. This makes the noise symmetric in $\bar{n}$, which is clearly wrong as shown in \fref{NoisePlotFull}. Only for small values of $\bar{n}$, when $(1-\bar{n})\sim 1$, the result matches the simulation. The noise in the diffusion part of the concentration, calculated via \aref{noise:promtoconc}, has no dependence on $\bar{n}$ any more, in contrast to our result in \eref{NoiseCFull} which is proportional to $1/(1-\bar{n})$. However, when a promoter is occupied for most of the time, it won't be able to `measure' new incoming particles, and thus doesn't make as much measurements as is possible given on-rate from the bulk. As a result, one would expect the noise to increase as $\bar{n}\to1$. Furthermore, de Ronde \cite{DeRonde2012} obtained precisely the same discrepancy for a receptor. Clearly, the extra $(1-\bar{n})$ factor in \eref{NoiseProTka} is wrong, and is most likely the result of a linearisation done in \cite{Tkacik2009}. 

The second difference is due to an ultraviolet cut-off needed in \cite{Tkacik2009}, resulting from a divergent integral. This divergence emerges in the Fourier back-transform analogous to \eref{FourierBackC}, and is the result of the DNA being represented as a cylinder with a zero radius. Physically, such a representation is clearly wrong, because the probability of a point particle finding a line is zero. To circumvent this problem, the upper boundary of the integral is cut-off at one over the smallest length scale in the system: $\pi/\sigma$. It is however not clear how well throwing away the higher frequencies compensates for the erroneous geometry. In our derivation (\eref{FDE_bulk}), we give the DNA a finite thickness, and the divergence in \eref{FourierBackC} does not occur. Although removing the cut-off gives a better quantitative agreement with simulation results, the approximation is qualitatively similar as shown in \fref{GaspervsJoris}.
\begin{figure}
\centering
\includegraphics[scale=.75]{GaspervsJoris}
\caption{\flabel{GaspervsJoris} Comparing the promoter noise level as defined in \eref{PowPro}, for a system with and without an ultraviolet cut-off. The cut-off was set at $\pi/\sigma$. The curve with cut-off is equal to Eq. 68 in \cite{Tkacik2009}, but without the extra $(1-\bar{n})$ factor in the diffusion part.}
\end{figure}

In the final act of his paper, Tka$\check{\rm c}$ik compares the concentration noise level of a promoter with a receptor; Fig. 2 in \cite{Tkacik2009}. Because his expression for the noise is independent of the occupancy, and because he assumes an infinite TF number, the noise has no minimum, similar to \fref{TableauxNoiseCmp} (d). Our result is therefore qualitatively very difference from that of Tka$\check{\rm c}$ik.



\subsection{Obtaining the correlation time by renormalizing the reaction rates}
As was shown by van Zon et al. \cite{VanZon2006}, we can describe our spatially resolved model by a well stirred model if we re-normalize the intrinsic binding and unbinding rates of the promoter $k_+$ and $k_-$ respectively. The idea is that the binding events of TF's to the promoter can be divided in two separate classes. Either the TF comes from the bulk without any prior knowledge of the position of the promoter, or the TF had just before dissociated from the promoter and started it's trajectory at the position of the promoter. The renormalized reaction rates (events per unit of time) $k'_{\rm on}$ and $k'_{\rm off}$ for the well stirred model are found by decreasing the intrinsic rates with all the binding events which are due to a spatial correlation between the TF's and the promoter:
\begin{equation}
 k'_{\rm on} = \frac{k_+}{1 + N_{\rm rb}} \quad \text{and} \quad k'_{\rm off} = \frac{k_-}{1 + N_{\rm rb}}.
\end{equation}
Here $N_{\rm rb}$ is the average number of rebindings before a TF escapes given that is starts in contact with the promoter. The off-rate $k_{\rm off}$ is obtained by demanding that the average occupancy of the promoter is unchanged by the renormalization. $N_{\rm rb}$ is calculated straightforwardly via
\begin{equation}
 N_{\rm rb} = \sum_{n=1}^{\infty} n \left( P_{\rm rebind} \right)^n P_{\rm escape} = \frac{1-P_{\rm escape}}{P_{\rm escape}},
\end{equation}
where $P_{\rm rebind}$ and $P_{\rm escape}$ are the splitting probabilities for either escaping or rebinding of a TF at contact with the promoter. The probability of a TF escaping is given by the $t \to \infty$ limit of the survival probability of a particle starting at contact 
\begin{equation}
 P_{\rm escape} = \lim_{t \to \infty} S_{\rm rad}(t|z_0) = S_{\rm rad}(\infty|z_0).
\end{equation}
The average number of bindings of a particle with the promoter before it escapes again is thus
\begin{equation}
 N_b = 1 + N_{\rm rb} = 1 + \frac{1 - P_{\rm escape}}{P_{\rm escape}} = P_{\rm escape}^{-1} = S_{\rm rad}(\infty|z_0)^{-1},
\end{equation}
and the renormalized rates are
\begin{equation}
 k'_{\rm on} = k_+ S_{\rm rad}(\infty|z_0) \quad \text{and} \quad  k'_{\rm off} = k_- S_{\rm rad}(\infty|z_0).
 \elabel{RenRates}
\end{equation}
The above on-rate is the rate at which TF's bind to the promoter in a system where none of the TF's have prior knowledge of the position of the promoter. Put differently, particles are placed at infinity after falling from the promoter. 

The correlation time for this system is
\begin{equation}
 \tau_c = \frac{1}{k'_{\rm on} \bar{c} + k'_{\rm off}} = \frac{1}{ (k_+ \bar{c} + k_- ) S_{\rm rad}(\infty|z_0) } = \frac{\tau_c^*}{S_{\rm rad}(\infty|z_0)}.
\end{equation}
Note that this expression for the correlation time is equivalent to that found in \eref{CT1} using the method by de Ronde et al. 

\subsubsection{Why do both methods give the same result?}
The important difference between the two methods, is that in the method of van Zon, a TF following a rebinding trajectory, will always find an unoccupied promoter when it returns. There is no hindrance by the other particles from the bulk. De Ronde, a priori, does allow for this by writing the survival probability of a promoter just after a TF dissociated from it, as a product of two survival probabilities; one for a single particle starting on the promoter, and one for the promoter in an equilibrated system (\eref{Def_Ssi}). As it turns out however, by using the low concentration limit, the time dependent behaviour at small time is coarse-grained out, in the same way as was done by van Zon.

To see this, we start from the relation used in the derivation of the correlation function \eref{SeqSsi}
\begin{equation}
 \pdiff{\mathscr{S}_{\rm rad}(t|{\rm eq})}{t} = -\bar{\xi} k_+ \, \mathscr{S}_{\rm rad}(t|z_0),
\end{equation}
which we Laplace transform, to find
\begin{equation}
 \hat{\mathscr{S}}_{\rm rad}(s|z_0) = \frac{1}{\bar{\xi}k_+} \left[ 1 - s\,\hat{\mathscr{S}}_{\rm rad}(s|{\rm eq})\right].
\end{equation}
Plugging in our small $\bar{\xi}$ approximation for $\hat{\mathscr{S}}_{\rm rad}(s|{\rm eq})$, \eref{Seq_approx}
\begin{equation}
 \lim_{\bar{\xi}\to 0} \hat{\mathscr{S}}_{\rm rad}(s|z_0) = \frac{1}{\bar{\xi} k_+} \left[ \frac{\bar{\xi}\,\hat{k}_{\rm rad}(s)}{1 + \bar{\xi}\,\hat{k}_{\rm rad}(s)}\right].
\end{equation}
In the limit of low $s$, which we use when calculating the correlation time, the lowest power of $s$ in $\hat{k}_{\rm rad}(s)$ dominates. For $t\to\infty$, $k_{\rm rad}(t)$ approaches the steady-state limit $k_{\rm ss}$: $\lim_{s\to0} \hat{k}_{\rm rad}(s) \sim k_{\rm ss}/s = k_+ S_{\rm rad}(\infty|z_0)/s$. Using this in the above equation, we can back-transform to the time domain
\begin{multline}
 \hat{\mathscr{S}}_{\rm rad}(s|z_0) = \frac{k_{\rm ss}}{k_+} \left[ \frac{1}{s + \bar{\xi}\,k_{\rm ss}} \right] \quad \xrightarrow{\mathscr{L}^{-1}} \\ \mathscr{S}_{\rm rad}(t|z_0) = S_{\rm rad}(\infty|z_0)\,{\rm exp}\left[ -\bar{\xi} k_+ S_{\rm rad}(\infty|z_0)\,t \right].
\end{multline}
Comparing this result with \eref{Def_Ssi}, we see that the rebinding trajectories are coarse-grained out by replacing $S_{\rm rad}(t|z_0)$ with $S_{\rm rad}(\infty|z_0)$. We are left with a Poisson decay of the survival probability, starting at the probability with which the particle starting at contact escapes. The reason this is a good approximation is due to the timescale seperation between particles comming from the bulk, and the rebinding of a particle starting on the promoter. As is shown in \aref{noise:Seq_approx}, \fref{SeqApprx}. Coarse graining out the non-Poissonian behaviour is exactly what van Zon does by renormalizing the reaction rates with the escape probability. Thus, in the low concentration limit of non-interacting TF's, the methods of van Zon and de Ronde to obtain the correlation time are equivalent.


