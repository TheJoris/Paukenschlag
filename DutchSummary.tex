\begin{center}
{\center \bf Samenvatting}
\end{center}

Voor het functioneren van een biologische cel, is het belangrijk dat deze kan regelen welke genen op het DNA naar eiwitten vertaald worden. Bepaalde eiwitten in deze cellen, de transcriptie factoren, kunnen de expressie van een bepaald gen 'aan' of 'uit' zetten. Door te binden aan een specifieke plek op het DNA, de zogenaamde promotor, kunnen deze eiwitten de expressie van een gen versterken of verhinderen. Op deze manier reageert de eiwit productie op veranderende transcriptie factor concentraties. Omdat er maar een paar van deze transcriptie factoren in een cel aanwezig zijn, en omdat het binden aan de promoter een toevalsproces is, variert de bezettingsgraad van de promoter door transcripie factoren sterk in de tijd. In deze scriptie word inzicht verkregen in hoe deze ruis afhangt van het gedrag van een transcriptie factor, zoals de diffusie in het cytoplasma en langs het DNA, voordat het bind met de promoter. 

Het onderzoek bestaat uit twee delen. Eerst wordt bekeken hoe een promoter op het DNA, en de diffusie van transcriptie factoren rond de promoter, gesimuleerd kan worden met een computer. Voor deze simulaties wordt het zogenaamde 'Green's Function Reaction Dynamics' (GFRD) gebruikt; een efficiente en preciese methode om diffusie en reacties van deeltjes te simuleren. Wij breiden GFRD zodanig uit dat ook promoters gesimuleerd kunnen worden.

Hierna wordt een model opgesteld voor het gedrag van een transcripie factor rond de promoter site. Op basis van dit model vinden we een exacte oplossing voor de ruis in de bezettingsgraad van een promoter over een tijdspanne $T$, en de ruis in de transcriptie factor concentratie die we uit de bezettingsgraad kunnen afleiden. We vinden dat de ruis in de afgeleide concentratie omgekeerd evenredig is met het aantal transcriptie factoren dat de promoter berijkt tijdens het interval $T$, maal de fractie van de tijd dat de promoter onbezet is. We vergelijken onze analytische uitdrukking met resultaten uit simulaties, en vinden dat we de geobserveerde ruis in de bezettingsgraad in deze simulaties uitstekend kunnen voorspellen.