\begin{center}
{\center \bf Samenvatting voor de leek}
\end{center}

Om te overleven moet een biologische cel voorturend regelen welke genen op zijn DNA naar eiwit vertaald worden. Sommige van deze eiwitten, de transcriptie factoren, kunnen door te binden aan een specifieke plek op het DNA, de zogenaamde promotor, een gen 'aan' of 'uit' zetten. Nu zijn er maar een paar van deze transcriptie factoren in een cel aanwezig, en zij voeren op weg naar de promotor een dronkaards beweging uit: een pad bestaande uit willekeurige, ongerichte stappen. Het gevolg is dat de bezetting van de promotor door een transcripie factor van minuut tot minuut sterk veranderd.  In deze scriptie wordt onderzocht hoe deze zogenaamde ruis afhangt van de kenmerken van de transcriptie factoren, de promotor en het DNA. 

Het onderzoek bestaat uit twee delen. Eerst wordt bekeken hoe een promotor,  en het pad van een transcriptie factor naar de promotor, gesimuleerd kan worden met een computer. Hiervoor wordt 'Green's Function Reaction Dynamics' gebruikt; een programma waarmee de dronkaards bewegingen van de deeltjes snel en precies wordt na gebootst. Voor dit onderzoek is het programma uitgebreid om ook promotors te simuleren. 

Hierna wordt een wiskundig model opgesteld dat de beweging van een transcripie factor rond de promotor beschrijft. Met dit model kunnen we  voorspellen hoe de bezetting van de promotor in een cel veranderd in de tijd. We vergelijken ons model met resultaten uit simulaties, en vinden dat de geobserveerde ruis uitstekend door ons model voorspeld wordt.