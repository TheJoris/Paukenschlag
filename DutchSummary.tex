\begin{center}
{\center \bf Samenvatting}
\end{center}

Voor het functioneren van een biologische cel, is het belangrijk dat deze kan regelen waarneer een gen op het DNA naar eiwit wordt vertaald. Bepaalde eiwitten in de cel, de transcriptie factoren, kunnen een gen 'aan' of 'uit' zetten. Door te binden aan een specifieke plek op het DNA, de zogenaamde promotor, kunnen deze eiwitten de expressie van een gen versterken of juist verhinderen. Dit stelt de cel in staat te reageren op fluctuaties in de transcriptie factor concentratie, en daarmee op veranderingen in zijn omgeving. Omdat er maar een paar van deze transcriptie factoren in een cel aanwezig zijn, en omdat het binden aan de promoter een toevalsproces is, varieert de bezettingsgraad van een promoter door transcripie factoren sterk in de tijd. In deze scriptie wordt onderzocht hoe deze zogenaamde ruis afhangt van de kenmerken van de transcriptie factoren, de promoter en het DNA. 

Het onderzoek bestaat uit twee delen. Eerst wordt bekeken hoe een promoter op het DNA, en de diffusie van transcriptie factoren naar deze promoter, gesimuleerd kan worden met een computer. Voor deze simulaties wordt 'Green's Function Reaction Dynamics' gebruikt; een algoritme dat op een slimme manier gebruikt dat de eiwit dichtheid zeer laag is, en zo het systeem effici\"ent en precies simuleert. Wij breiden het algoritme uit zodat ook promoters gesimuleerd kunnen worden.

Hierna wordt een model opgesteld voor het gedrag van een transcripie factor rond de promoter. Op basis van dit model vinden we een exacte oplossing voor de ruis in de bezettingsgraad van een promoter, en de ruis in de transcriptie factor concentratie die we uit de bezettingsgraad kunnen afleiden. We vinden dat de ruis in de afgeleide concentratie toeneemt met de gemiddelde bezettingsgraad, en dat de ruis sterk afhankelijk is van de tijd die een transcripie factor op het DNA verblijft. We vergelijken onze analytische uitdrukking met resultaten uit simulaties, en vinden dat de geobserveerde ruis in de bezettingsgraad uitstekend door ons model voorspelt wordt.