% \setcounter{figure}{0}
% \setcounter{equation}{0}
% \setcounter{table}{0}
% \renewcommand{\theequation}{SI-\arabic{equation}}
% \renewcommand{\thefigure}{SI-\arabic{figure}}\renewcommand{\figurename}{Figure}
% \renewcommand{\thetable}{SI-\arabic{table}}
% \setcounter{section}{0}
% \renewcommand{\thesection}{SI-\Alph{section}}
% \setcounter{subsection}{0}
% \renewcommand{\thesubsection}{SI-\arabic{subsection}}
% \setcounter{subsubsection}{0}
% \renewcommand{\thesubsubsection}{SI-\arabic{subsubsection}}

\section{Appendix: Noise in transcriptional regulation}

\subsection{\alabel{noise:corr_time} The correlation time}
The time-average of an observable $A\br{t}$\cite{Frenkel2002} is defined as
\begin{align}
\elabel{si:time_av_mean}
A_T=\frac{1}{T}\int_0^T dt A\br{t}
\end{align}
and the variance of the time-averaged mean $\sigma^2_{A_T}$ is
\begin{align}
\sigma^2_{A_T}&\br{T}=\avg{A^2_T}-\br{\frac{1}{T}\int \avg{A\br{t}}dt}^2\\
&=\frac{1}{T^2}\int_0^T  \int_0^T dt dt' \avg{A\br{t}A\br{t'}}-\avg{A_T}^2\\
\elabel{si:time_av_var2}
&=\frac{1}{T^2}\int_0^T dt \int_{-t}^{T-t} d\tau \avg{A\br{0}A\br{t'-t}}-\avg{A_T}^2\\
\elabel{si:time_av_var3}
&=\frac{1}{T^2}\int_0^T dt\int_{-t}^{T-t} d\tau \avg{A\br{0}A\br{\tau}}-\avg{A}^2,
\end{align}
where for \eref{si:time_av_var2} the process is assumed to be stationary and we define $\tau=t'-t$.
The correlation function for this observable $A\br{t}$ is defined as
\begin{align}
\elabel{si:time_corr_ens}
C\br{t,t'}&=\avg{\br{A\br{t'}-\avg{A}}\br{A\br{t}-\avg{A}}}\\
\elabel{si:time_corr_ens_1}
C\br{\tau}&=\avg{\br{A\br{\tau}-\avg{A}}\br{A\br{0}-\avg{A}}}\\
\elabel{si:time_corr_ens_2}
&=\avg{A\br{0}A\br{\tau}}-\avg{A}^2
\end{align}

Substitution of \eref{si:time_corr_ens_2} into \eref{si:time_av_var3} leads to
\begin{align}
\sigma^2_{A_T}\br{T}&=\frac{1}{T^2}\int_0^T dt \int_{-t}^{T-t} d\tau C\br{\tau}+\frac{1}{T}\int_0^T \avg{A}^2-\avg{A}^2\\
\elabel{si:time_av_var4}
&=\frac{1}{T^2}\int_0^T dt \int_{-t}^{T-t} d\tau C\br{\tau}.
\end{align}
In the limit of small $T$ and large $T$ a solution for \eref{si:time_av_var4} can be obtained
\begin{align}
\elabel{si:limT_0_corr}
&\sigma^2_{A_T}\br{0}=C\br{0}=\sigma_A^2\\
\elabel{si:limT_infty_corr}
&\lim_{T\to\infty}\approx \lim_{T\gg \tau_A}=\sigma^2_{A_T}\br{T\gg \tau_A}=\frac{1}{T^2}\int_0^T dt \int_{-\infty}^{\infty} d\tau
C\br{\tau}\nonumber\\
&\quad\qquad\qquad=\frac{2\sigma_A^2\tau_A}{T},
\end{align}
where we used the fact that $\lim_{\tau\gg\tau_A} C\br{\tau}=0$. We introduced the correlation time $\tau_A$ in \eref{si:limT_infty_corr}
which is commonly defined as
\begin{align}
\elabel{si:corr_time}
\tau_A\equiv \frac{1}{\sigma^2_A}\int_0^\infty C\br{\tau} d\tau
\end{align}
As an example, for a process $A\br{t}$ with exponential waiting times, $C\br{\tau}\propto e^{-k \tau}$ and $\int C\br{\tau}d\tau \propto
k^{-1} \sim \tau_A$.

The power spectrum ($P_A\br{\omega}$) and correlation function are related through the Fourier Transform
\begin{align}
\elabel{si:rel_pow_lap}
C_A\br{\tau}&=\frac{1}{\sqrt{2\pi}}\int_{-\infty}^\infty d\omega P_A\br{\omega}e^{i\omega\tau}\\
P_A\br{\omega}&=\frac{1}{\sqrt{2\pi}}\int_{-\infty}^\infty d\tau C_A\br{\tau}e^{-i\omega\tau}
\end{align}
such that
\begin{align}
C_A\br{0}&=\frac{1}{\sqrt{2\pi}}\int_{-\infty}^\infty d\omega P_A\br{\omega}=\sigma_A^2\\
P_A\br{0}&=\frac{1}{\sqrt{2\pi}}\int_{-\infty}^\infty d\tau C_A\br{\tau}\equiv 2\sigma_A^2\tau_A=T\sigma^2_{A_T}
\end{align}
where $\tau_A$ is the correlation time (see \eref{si:corr_time}).

Next, we note the relation between the Laplace transform and the Fourier transform
\begin{align}
\elabel{si:FT_of_corr}
{\cal F}&\br{C_A\br{\tau}}=P_A\br{\omega}\\
&={\rm Re}\sqbr{{\cal L}\br{C_A\br{t}}}+{\rm Re}\sqbr{{\cal L}\br{C_A\br{t}}^{\dagger}}\\
&={\rm Re}\sqbr{\hat{C}_A\br{s=i\omega}}+{\rm Re}\sqbr{\hat{C}_A\br{s=-i\omega}}\\
&=2{\rm Re}\sqbr{\hat{C}_A\br{s=i\omega}}.
\end{align}
The correlation time is therefore related to the Laplace transform of the correlation function by
\begin{align}
\elabel{si:laplace_corr_func_corr_time}
&\sigma_A^2\tau_A=P_A\br{0}\\
&={\rm Re}\sqbr{\hat{C}\br{s=i\omega}}_{\omega=0}+{\rm Re}\sqbr{\hat{C}\br{s=-i\omega}}_{\omega=0}\\
&=2{\rm Re}\sqbr{\hat{C}\br{s=i\omega}}_{\omega=0}
\end{align}

\subsection{\alabel{noise:limits_Laplace} Limits in Laplace space}
Here we show important quantities in the time domain of a certian function, by taking limits in the Laplace domain. We can find the time integral of an arbitrary function by taking the limit
\begin{align}
\lim_{s \rightarrow 0} \hat{A}(s) = \int_0^\infty A(t) dt.
\elabel{si:int_vs_lap}
\end{align}
The long time limit of a function is given by
\begin{align}
\lim_{s \rightarrow 0} s \, \hat{A}(s) = \lim_{t \rightarrow \infty} A(t).
\elabel{si:longt_in_lap}
\end{align}
In the same spirit
\begin{align}
\lim_{s \rightarrow 0} s^2 \, \hat{A}(s) = \lim_{t \rightarrow \infty} \partial_t A(t).
\elabel{si:longt_in_lap2}
\end{align}

\subsection{\alabel{noise:backward_eqn} The backward Smoluchowski equation}
\ldots

\subsection{\alabel{noise:Seq_approx} Approximating the $\mathscr{S}_{\rm irr}(t|{\rm eq})$ function}
We want to approximate the equation for $\mathscr{S}_{\rm irr}(t|{\rm eq})$
\begin{equation}
 \mathscr{S}_{\rm irr}(t|{\rm eq}) = e^{-\bar{\xi}\int_0^t k_{{\rm irr}}(t') dt'},
\end{equation}
such that it can be Laplace transformed, and has correct behaviour in the long time limit. We will use the approximation in the limit of small $s$, and thus for large $t$. Because the argument of the exponential will diverge for large $t$, a small argument expansion will not work. We can, however, make a large time approximation by replacing $k_{{\rm irr}}(t)$ with $k_{{\rm irr}}(t \rightarrow \infty)$. Since a particle starting at the promoter will be able to escape, the large time limit of $k_{{\rm irr}}>0$. The survival probability becomes  
\begin{equation}
 \lim_{t \rightarrow \infty} \mathscr{S}_{\rm irr}(t|{\rm eq}) = e^{-\bar{\xi} k_{{\rm irr},\infty}\,t}.
\end{equation}
Laplace transforming gives
\begin{equation}
 \lim_{s \rightarrow 0} \hat{\mathscr{S}}_{\rm rad}(s|eq) = \frac{1}{s} \, \frac{1}{1 + \bar{\xi} \hat{k}_{\rm rad}(s)}.
\end{equation}

\subsection{\aref{noise:delta_int} Performing the integral over $\delta{r-\sigma}$}


